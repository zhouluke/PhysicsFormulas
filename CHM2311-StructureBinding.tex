% !TeX spellcheck = en_US
\documentclass[10pt,landscape]{article}
\usepackage{multicol}
\usepackage{calc}
\usepackage{ifthen}
\usepackage[landscape]{geometry}
\usepackage{hyperref}
\usepackage{amsmath, amsfonts}
\usepackage{mathtools}
\usepackage {graphicx}
\usepackage {textcomp}

\usepackage{siunitx}
\usepackage{physics}

% English spacing
\usepackage{icomma}
\frenchspacing
\usepackage[english]{babel}
 \sisetup{output-decimal-marker = {.}}
 
 
% To make this come out properly in landscape mode, do one of the following
% 1.
%  pdflatex latexsheet.tex
%
% 2.
%  latex latexsheet.tex
%  dvips -P pdf  -t landscape latexsheet.dvi
%  ps2pdf latexsheet.ps



% This sets page margins to .5 inch if using letter paper, and to 1cm
% if using A4 paper. (This probably isn't strictly necessary.)
% If using another size paper, use default 1cm margins.
\ifthenelse{\lengthtest { \paperwidth = 11in}}
	{ \geometry{top=.5in,left=.5in,right=.5in,bottom=.5in} }
	{\ifthenelse{ \lengthtest{ \paperwidth = 297mm}}
		{\geometry{top=1cm,left=1cm,right=1cm,bottom=1cm} }
		{\geometry{top=1cm,left=1cm,right=1cm,bottom=1cm} }
	}

% Turn off header and footer
\pagestyle{empty}
 

% Redefine section commands to use less space
\makeatletter
\renewcommand{\section}{\@startsection{section}{1}{0mm}%
                                {-1ex plus -.5ex minus -.2ex}%
                                {0.5ex plus .2ex}%x
                                {\normalfont\large\bfseries}}
\renewcommand{\subsection}{\@startsection{subsection}{2}{0mm}%
                                {-1explus -.5ex minus -.2ex}%
                                {0.5ex plus .2ex}%
                                {\normalfont\normalsize\bfseries}}
\renewcommand{\subsubsection}{\@startsection{subsubsection}{3}{0mm}%
                                {-1ex plus -.5ex minus -.2ex}%
                                {1ex plus .2ex}%
                                {\normalfont\small\bfseries}}
\makeatother

% Don't print section numbers
\setcounter{secnumdepth}{0}


\setlength{\parindent}{0pt}
\setlength{\parskip}{0pt plus 0.5ex}

\newcommand{\extraline}{\vspace{1em}}
\newcommand{\halfline}{\vspace{0.5em}}
\newcommand{\deriv}{\ensuremath{\frac{d}{dx}}}
\newcommand{\tableindent}{\hspace{1.5em}}
\newcommand{\uvec}[1]{\ensuremath{{\hat{#1}}}}
% -----------------------------------------------------------------------

\begin{document}

\raggedright
\footnotesize
\begin{multicols}{3}

% multicol parameters
% These lengths are set only within the two main columns
%\setlength{\columnseprule}{0.25pt}
\setlength{\premulticols}{1pt}
\setlength{\postmulticols}{1pt}
\setlength{\multicolsep}{1pt}
\setlength{\columnsep}{2pt}

\begin{center}
     \Large{\textbf{Structure \& Bonding}} \\
     \small{Luke Zhou $\cdot$ CHM 2311 $\cdot$ Winter 2022}
\end{center}

\section{Atomic Structure \& Quantum Theory}

Wavelength, frequency and the speed of light
\[ c = \lambda\nu  = \SI{2.998E8}{\meter/\second}  \]

Energy quantization
\[E = \frac{hc}{\lambda} = h\nu  \qquad
 h = \SI{6.626E-34}{\joule\cdot\second}  \]

Energy emitted by a hydrogen atom: Bohr model
 \[ \Delta E = E_{h} - E_{l} = R_H \left(  \frac{1}{n_l^2} - \frac{1}{n_h^2} \right) \] 
 \[ R_H =  \SI{2.179E-18}{\joule} \]

Hydrogen emission spectra series \\
\begin{tabular}{@{\tableindent}ll@{}}
	Lyman series & $n_l = 1$ \\
	Balmer series & $n_l = 2$ \\
	Paschen series & $n_l = 3$ \\
	Brackett series & $n_l = 4$ \\
	Pfund series & $n_l = 5$ \\
	Humphreys series & $n_l = 6$ \\
\end{tabular}
\extraline

Wave-Particle Duality  
\[  mv = \frac{h}{\lambda} \]


Heisenberg Uncertainty Principle
\[  \Delta x \Delta p \geq \frac{h}{4\pi} \]

Schrödinger Wave Equation
\[ H\Psi = E\Psi  \]
\[ \left[ \frac{-h^2}{8\pi^2m} +
V(x,y,z)\right] \Psi(x,y,z) = E\Psi(x,y,z)  \]
	
Particle in a box
\[ \Psi(x) = \sqrt{\frac{2}{a}} \sin\left( \frac{n\pi x}{a} \right)  \qquad n \in \mathbb{N} \]
\[ E = \frac{n^2h^2}{8ma^2} \]

Particle on a ring
\[ \Psi(x) = B \exp \left(\frac{in\pi x}{L} \right)
\qquad n \in \mathbb{Z}, L = n\lambda  \]
\[ E = \frac{h^2}{8mL^2} (2n)^2 \]

Hydrogenic wave functions
\[ \Psi(r,\theta,\phi) = R(r) Y(\theta,\phi) \]

Quantum numbers \\
\extraline
\begin{tabular}{@{\tableindent}lp{1.5cm}lp{2.3cm}@{}}
$n$ & Principal & $1, 2, 3, 4, \ldots $ & \scriptsize{orbital size} \\
$l$ & Angular momentum & $0, 1, 2 \dots  n-1 $ & \scriptsize{orbital shape} \\
$m_l$ & Magnetic & $0, \pm 1, \pm 2, \ldots \pm l $ & \scriptsize{orbital orientation} \\
$m_s$ & Spin & $-\frac{1}{2}, +\frac{1}{2} $ &  \\
\end{tabular}

\extraline
\tableindent $l$ can also be described with the letters $s, p, d, f, g, \ldots$
\extraline

Radial \& angular nodes

\begin{tabular}{@{\tableindent}llp{2.4cm}@{}}
\# of radial nodes & $n-l-1$ & $R(r) = 0$ \scriptsize{(spheres)}\\
\# of angular nodes & $l$ &  $Y(\theta,\phi)=0$ \scriptsize{(planes, cones)} \\
Total \# of nodes & $n-1$ & 
\end{tabular}
\extraline

Electron configurations

\begin{tabular}{@{\tableindent}p{2.5cm}p{4.5cm}@{}}
Aufbau Principle & Fill lower energy levels first: 1s, 2s, 2p, 3s, 3p, \textbf{4s}, \textbf{3d}, 4p, 5s, 4d, 5p, \ldots\\
Pauli Exclusion Principle & No two e- can have the same set of quantum numbers. \\
Hund's Rule & Fill degenerate levels separately w/ spins parallel -- minimize energies below \\
\end{tabular}
\extraline

Energies to be minimized

\begin{tabular}{@{\tableindent}ll@{}}
	Coulombic Energy of Repulsion & \textdownarrow{} e- pairing = \textdownarrow{} $E$ \\
	Exchange Energy & \textuparrow{} spin = \textdownarrow{} $E$
\end{tabular}
\extraline


Magnetism

\begin{tabular}{@{\tableindent}lp{5cm}@{}}
Paramagnetism & Attracted by a magnetic field; elements w/ unpaired e- \\
Dimagnetism & Slightly repelled by a magnetic field; elements w/ no unpaired e-
\end{tabular}

Transition Metals
\begin{itemize}
\item In ionization, e- are first removed from orbitals w/ density farthest from nucleus; i.e. from $ns$ before $(n-1)s$
\item Half- and fully-filled shells minimize exchange energy \& e- pairing: e.g., Cr ([Ar]4s$^1$3d$^{5}$) \& Cu  ([Ar]4s$^1$3d$^{10}$) 
\item All $M^{2+}$ ions are 3d-only; e.g.,  Fe$^{2+}$ ([Ar]$3d^{6}$) 
%\& Fe ([Ar]$4s^23d^{6}$)
\end{itemize}  

\extraline

Effective nuclear charge
\[ Z_\text{eff} = Z_\text{actual} - S \]

Slater's screening constant \& rules
\begin{enumerate}
\item List orbitals in groups, merging s \& p: (1s) (2s, 2p) (3s, 3p) (3d) (4s, 4p) (4d) (5s, 5p), \ldots 
\item For e- in $ns$ or $np$ orbitals:

	\begin{enumerate}
	\item Each e- in the same $ns$ or $np$ (i.e. same grouping) as the one of interest contributes 0.35 (except 1s: 0.30)
	\item Each in the $(n-1)$ level: 0.85
	\item Each in the $(n-2)$ levels or lower: 1.00
	\end{enumerate}

\item For e- in $nd$ or $nf$ orbitals:

	\begin{enumerate}
	\item Each e- in the same $nd$ or $nf$  orbital as the e- of interest contributes 0.35
	\item Each in the groupings to the left: 1.00
	\end{enumerate}

\end{enumerate}



Periodic trends

\begin{center}
\includegraphics[width=8cm]{./chm2311/periodic-table-trends.png}
\end{center}
 
\begin{itemize} 
\item Atomic radius: size of outermost orbital containing e-. Larger $n$ or higher $Z_\text{eff}$ = lower radius.
\item Ionization energy: for removal of an e-. Always endothermic. Larger atomic radius  = lower IE.
\item Second IE is much higher than the first.
\item Electron affinity: energy in adding an e- to an isolated atom. Usually exothermic (i.e. negative). Larger atomic radius = less attraction to nucleus = less-negative EA.
\item Second EA is always endothermic. 
\item Electronegativity: ability of a bonded atom to attracted shared electrons. EN is relative.
\end{itemize}

\hrulefill


\section{Constants}

Planck's constant
\[ h = \SI{6.626E-34}{\joule\cdot\second} \]
 
 Rydberg constant for hydrogen
 \[ R_H =  \SI{2.179E-18}{\joule} \]
 
Electron mass
\[  m_e = \SI{9,11E-31}{\kilogram} \]

Speed of light
\[ c = \SI{2.998E8}{\meter/\second}  \]

Bohr radius
\[ a_0 = \SI{52.9}{\pico\metre} \]

Conversions
\[ \SI{1}{\metre} = 10^9 \, \si{\nano\metre} = 10^{12} \, \si{\pico\metre}\]
\[ \SI{1}{\joule} = \SI{1}{\kilogram\cdot\metre^2/\sec^2}  \]


\hrulefill


\scriptsize

\href{https://github.com/zhouluke/PhysicsFormulas}{Typography}  \copyright\ 2022 Luke Zhou. \\
Based on the course CHM 2311 Winter 2022 at the University of Ottawa \\
\href{http://wch.github.io/latexsheet/}{Document template}  \copyright\ 2014 Winston Chang.


\end{multicols}
\end{document}
