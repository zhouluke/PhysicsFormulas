% !TeX spellcheck = fr_FR
\documentclass[10pt,landscape]{article}
\usepackage{multicol}
\usepackage{calc}
\usepackage{ifthen}
\usepackage[landscape]{geometry}
\usepackage{hyperref}
\usepackage{amsmath}
%\usepackage[h]{esvect}

\usepackage{siunitx}
\usepackage{physics}

% French spacing
\usepackage{icomma}
\frenchspacing
\usepackage[french]{babel}
 \sisetup{output-decimal-marker = {,}}
 
 
% To make this come out properly in landscape mode, do one of the following
% 1.
%  pdflatex latexsheet.tex
%
% 2.
%  latex latexsheet.tex
%  dvips -P pdf  -t landscape latexsheet.dvi
%  ps2pdf latexsheet.ps



% This sets page margins to .5 inch if using letter paper, and to 1cm
% if using A4 paper. (This probably isn't strictly necessary.)
% If using another size paper, use default 1cm margins.
\ifthenelse{\lengthtest { \paperwidth = 11in}}
	{ \geometry{top=.5in,left=.5in,right=.5in,bottom=.5in} }
	{\ifthenelse{ \lengthtest{ \paperwidth = 297mm}}
		{\geometry{top=1cm,left=1cm,right=1cm,bottom=1cm} }
		{\geometry{top=1cm,left=1cm,right=1cm,bottom=1cm} }
	}

% Turn off header and footer
\pagestyle{empty}
 

% Redefine section commands to use less space
\makeatletter
\renewcommand{\section}{\@startsection{section}{1}{0mm}%
                                {-1ex plus -.5ex minus -.2ex}%
                                {0.5ex plus .2ex}%x
                                {\normalfont\large\bfseries}}
\renewcommand{\subsection}{\@startsection{subsection}{2}{0mm}%
                                {-1explus -.5ex minus -.2ex}%
                                {0.5ex plus .2ex}%
                                {\normalfont\normalsize\bfseries}}
\renewcommand{\subsubsection}{\@startsection{subsubsection}{3}{0mm}%
                                {-1ex plus -.5ex minus -.2ex}%
                                {1ex plus .2ex}%
                                {\normalfont\small\bfseries}}
\makeatother

% Don't print section numbers
\setcounter{secnumdepth}{0}


\setlength{\parindent}{0pt}
\setlength{\parskip}{0pt plus 0.5ex}

\newcommand{\extraline}{\vspace{1em}}
\newcommand{\halfline}{\vspace{0.5em}}
\newcommand{\deriv}{\ensuremath{\frac{d}{dx}}}
\newcommand{\tableindent}{\hspace{1.5em}}
\newcommand{\uvec}[1]{\ensuremath{{\hat{#1}}}}
% -----------------------------------------------------------------------

\begin{document}

\raggedright
\footnotesize
\begin{multicols}{3}

% multicol parameters
% These lengths are set only within the two main columns
%\setlength{\columnseprule}{0.25pt}
\setlength{\premulticols}{1pt}
\setlength{\postmulticols}{1pt}
\setlength{\multicolsep}{1pt}
\setlength{\columnsep}{2pt}

\begin{center}
     \Large{\textbf{Électricité et magnétisme}} \\
     \small{Luke Zhou $\cdot$ PHY 2723 $\cdot$ Hiver 2021}
\end{center}

\section{Électrostatique}

Loi de Coulomb : force électrique [\si{\newton}]
 \[ \vec{F} = q\vec{E}(\vec{r}) \] %= \frac{1}{4\pi\epsilon} \frac{kq_1q_2}{|r|^2} \uvec{r}  \]

Charge  [\si{\coulomb}]
\begin{gather*} 
dq = \rho_l(\vec{r}) \,dl  \qquad  dq = \rho_s(\vec{r}) \,ds  \qquad dq = \rho_v(\vec{r}) \,dv  \\
 \textstyle Q = \int dq 
\end{gather*}

Conventions de notation  \\
\halfline
\begin{tabular}{@{\tableindent}ll@{}}
	Avec $'$ & Point de source\\
	Sans $'$ & Point d'observation \\
\end{tabular}

\extraline
Champ électrique [$\si{\newton/\coulomb}$, $\si{\volt/\metre}$]
\begin{gather*}
\vec{E}(\vec{r}) = \frac{1}{4\pi\epsilon} \int \frac{dq}{| \vec{r} - \vec{r}\,' |^2 }  \\
%
 \vec{E}(\vec{r}) = \frac{1}{4\pi\epsilon} \int_{l} \frac{\rho_l(\vec{r}\,')}{| \vec{r} - \vec{r}\,' |^3 } (\vec{r} - \vec{r}\,' )\,dl'  \\
%
\vec{E}(\vec{r}) = \frac{1}{4\pi\epsilon} \int_{s} \frac{\rho_s(\vec{r}\,')}{| \vec{r} - \vec{r}\,'|^3 } (\vec{r} - \vec{r}\,' )\,ds' \\
%
\vec{E}(\vec{r}) = \frac{1}{4\pi\epsilon} \int_{v} \frac{\rho_v(\vec{r}\,')}{| \vec{r} - \vec{r}\,' |^3 } (\vec{r} - \vec{r}\,' )\,dv'  
\end{gather*}
	
Densité de flux électrique (Champ $\vec{D}$)  [$\si{\coulomb/\metre^2}$]
\[ \vec{D} = \epsilon\vec{E}
\qquad \nabla \cdot \vec{D} = \rho_v \]

Flux électrique [\si{\coulomb}]
\[  \Psi = \int_{s} \vec{D} \cdot d\vec{s} = \epsilon \int_{s} \vec{E} \cdot d\vec{s} \]

Théorème de Gauss
\[ \oint_s \vec{E}\cdot d\vec{s} = \frac{Q_\text{int}}{\epsilon}
\qquad  \Psi = \oint_s \vec{D}\cdot d\vec{s} = Q_\text{int} \]

Potentiel [$\si{\volt} = \si{\newton\cdot\metre/\coulomb} = \si{\joule/\coulomb}$]
\begin{gather*}
V(\vec{r}) = \frac{1}{4\pi\epsilon} \int \frac{dq}{| \vec{r} - \vec{r}\,' | }  \qquad
V(\vec{r}) = \frac{1}{4\pi\epsilon} \int_{l} \frac{\rho_l(\vec{r}\,')}{| \vec{r} - \vec{r}\,' | } \,dl'  \\
V(\vec{r}) = \frac{1}{4\pi\epsilon} \int_{s} \frac{\rho_s(\vec{r}\,')}{| \vec{r} - \vec{r}\,'| } \,ds'  \qquad
V(\vec{r}) = \frac{1}{4\pi\epsilon} \int_{v} \frac{\rho_v(\vec{r}\,')}{| \vec{r} - \vec{r}\,' | } \,dv'  
\end{gather*}

Potentiel et champ électrique
\begin{gather*}
\Delta V = V_B - V_A  = -\int_A^B \vec{E}\cdot d\vec{l} \qquad
 \vec{E} = -\nabla V
\end{gather*}

Équilibre électrostatique $\Rightarrow \vec{E}$ conservatif  $\Rightarrow \nabla \times \vec{E} = 0$
\[  \oint_C \vec{E}\cdot d\vec{l}  = 0 \qquad \text{(électrostatique)}\]

Travail [\si{\joule}] effectué par une force extérieure en déplaçant $q$ contre un champ $\vec{E}$
\[ W_{A\to B} = W_\text{nc} = \Delta U_E = -q\int_{A}^{B} \vec{E}\cdot d\vec{l} = q\Delta V \]

Dipôle électrique
\[ V(\vec{r}) = \frac{q}{4\pi\epsilon} \frac{d\cos\theta}{r^2} =
\frac{\vec{p}\cdot\uvec{r}}{4\pi\epsilon|\vec{r}|^2} =
\frac{\vec{p}\cdot\vec{r}}{4\pi\epsilon|\vec{r}|^3} \]

Moment dipolaire [$\si{\coulomb\cdot\meter}$]
\[ \vec{p} = qd\cdot \uvec{p} \qquad \text{($\uvec{p}: q- \to q+$;  $\uvec{p}\parallel\uvec{z}$ souvent)} \]

\halfline
Conditions aux limites entre un conducteur et un diélectrique  (ou le vide)\\
\halfline
\begin{tabular}{@{\tableindent}ll@{}}
	Pour le diélectrique & $ E_n = {\rho_s}/{\epsilon} \qquad E_t = 0 $ \\
	Pour le conducteur  & $\vec{E} = 0 \qquad\qquad \rho_v = 0 $ \\
\end{tabular}

\extraline
Conditions aux limites entre deux diélectriques (ou un diélectrique et le vide)
\begin{gather*}
\vec{E}_{1t} =\vec{ E}_{2t}  \\
 (\vec{D}_1 - \vec{D}_2) \cdot \uvec{n} = \rho_s  \qquad (\rho_s=0 \Rightarrow \epsilon_1 \vec{E}_{1n} = \epsilon_2 \vec{E}_{2n})
\end{gather*}

\halfline
Diélectriques linéaires
\[ \vec{D} = \epsilon_0\vec{E} + \vec{P} = \epsilon_0(1+\chi)\vec{E} = \epsilon_0\epsilon_r\vec{E} = \epsilon\vec{E}  \]
\begin{tabular}{@{\tableindent}ll@{}}
	Polarisation  & $\vec{P} \equiv d\vec{p}/dv \quad (\text{$v$: volume})$ \\
	  & $\vec{P} = \epsilon_0\chi\vec{E}$ \\
	Susceptibilité diélectrique  & $\chi$ \\
	Permittivité relative &  $\epsilon_r=1+\chi$ \\
	Permittivité & $\epsilon = \epsilon_0\epsilon_r$
\end{tabular}

\extraline
Condensateurs  \\
\halfline
\begin{tabular}{@{\tableindent}ll@{}}
	Capacitance, général [\si{\farad}] & $\displaystyle C \equiv {Q}/{\Delta V}  > 0 $ \\
	Potentiel  & $\Delta V = V_+ - V_-$ \\
	Direction du champ $\vec{E}$ & $\uvec{E}: +Q \to -Q$ \\
	Électrodes parallèles & $C = {A\epsilon}/{d}$ \\
	 & $E = {\sigma}/{\epsilon_0}$ \\
	 & $E =  -{\Delta V}/{d}$ \\
	Condensateurs en parallèle  & $C_\text{éq} = C_1 + C_2 + \dots$ \\
	Condensateurs en série & $\frac{1}{C_\text{éq}} = \frac{1}{C_1} + \frac{1}{C_2} + \dots $\\
\end{tabular}


\halfline
Énergie emmagasinée dans un champ électrique
\[ W_E = \frac{1}{2} \int_v  \rho_v(\vec{r})  \, V(\vec{r}) \, dv 
= \frac{1}{2} \int_v  \vec{D}\cdot\vec{E} \, dv 
= \frac{1}{2} \int_v  \epsilon E^2 \, dv 
\frac{}{}\]
\[ W_E = \frac{1}{2} QV = \frac{1}{2} CV^2 = \frac{1}{2}\frac{Q^2}{C} \quad\text{(condensateur)} \]

\halfline
\begin{tabular}{@{}ll@{}}
	Équation de Poisson ($\epsilon$ const) & $\nabla^2 V = -{\nabla\rho_v}/{\epsilon}$ \\
	Équation de Laplace ($\rho_v=0$) & $\nabla^2 V = 0$ \\ 
\end{tabular}


\extraline
\hrulefill



\section{Systèmes de coordonnées}

\subsection{Coordonnées cartésiennes $(x,y,z)$}

Gradient
\[ \nabla V = \frac{\partial V}{\partial x}\uvec{x} + \frac{\partial V}{\partial y}\uvec{y} + \frac{\partial V}{\partial z}\uvec{z} \]

Laplacien
\[ \nabla^2 V = \frac{\partial^2 V}{\partial x^2} + \frac{\partial^2 V}{\partial y^2} + \frac{\partial^2 V}{\partial z^2} \]

Rotationnel
\[ \nabla\times \vec{E} =
\left( \frac{\partial E_z}{\partial y} - \frac{\partial E_y}{\partial z}\right)\uvec{x}
+ \left( \frac{\partial E_x}{\partial z} - \frac{\partial E_z}{\partial x}\right)\uvec{y}
+ \left( \frac{\partial E_y}{\partial x} - \frac{\partial E_x}{\partial y}\right)\uvec{z} \]

Éléments différentiels
\begin{gather*}
	d\vec{l} = dx\,\uvec{x} + dy\,\uvec{y} + dz\,\uvec{z} \\
	d\vec{s} = dy\,dz\,\uvec{x} + dx\,dz\,\uvec{y} + dx\,dy\,\uvec{z} \\
	dv = dx\,dy\,dz
\end{gather*}

\subsection{Coordonnées cylindriques $(\rho, \phi, z)$}

Restrictions: $\phi\in[0,2\pi]$
\[ x = \rho\cos\phi \qquad y = \rho\sin\phi \qquad z=z  \]
\[ \rho = \sqrt{x^2 + y^2} \qquad \phi = \arctan\left(\frac{y}{x}\right) \]
%
\[
\begin{bmatrix}
	A_x \\ A_y \\ A_z
\end{bmatrix}
=
\begin{bmatrix}
	\cos\phi & -\sin\phi & 0 \\
	\sin\phi & \cos\phi & 0 \\
	0 & 0 & 1 \\
\end{bmatrix}
\begin{bmatrix}
	A_\rho \\ A_\phi \\ A_z
\end{bmatrix}
\]
\[
\begin{bmatrix}
	A_\rho \\ A_\phi \\ A_z
\end{bmatrix}
=
\begin{bmatrix}
	\cos\phi & \sin\phi & 0 \\
	-\sin\phi & \cos\phi & 0 \\
	0 & 0 & 1 \\
\end{bmatrix}
\begin{bmatrix}
	A_x \\ A_y \\ A_z
\end{bmatrix}
\]

\extraline
Gradient
\[ \nabla V = \frac{\partial V}{\partial \rho}\uvec{\rho} + \frac{1}{\rho}\frac{\partial V}{\partial \phi}\uvec{\phi} + \frac{\partial V}{\partial z}\uvec{z} \]

Laplacien
\[ \nabla^2 V = 
\frac{1}{\rho} \frac{\partial}{\partial \rho} \left( \rho\frac{\partial V}{\partial \rho} \right)
+ \frac{1}{\rho^2}\frac{\partial^2 V}{\partial \phi^2}
+ \frac{\partial^2 V}{\partial z^2} \]

Éléments différentiels
\begin{gather*}
d\vec{l} = d\rho\,\uvec{\rho} + \rho\,d\phi\,\uvec{\phi} + dz\,\uvec{z} \\
d\vec{s} = \rho\,d\phi\,dz\,\uvec{\rho} + d\rho\,dz\,\uvec{\phi} + \rho\,d\rho\,d\phi\,\uvec{z} \\
dv = \rho\,d\rho\,d\phi\,dz
\end{gather*}

\subsection{Coordonnées sphériques $(r, \theta, \phi)$}

Restrictions: $r\geq 0,  \theta\in[0, \pi], \phi\in[0, 2\pi]$ \\
$\phi$ est l'angle formé dans le plan $xy$. \\
$\theta$ est l'angle entre le vecteur de position et l'axe des $z$.
%
\[ x = r\sin\theta\cos\phi \qquad y =  r\sin\theta\sin\phi \qquad z= r\cos\theta  \]
\[ r = \sqrt{x^2 + y^2 + z^2} \qquad \theta = \arccos\left(\frac{z}{r}\right) \qquad \phi = \arctan\left(\frac{y}{x}\right) \]
%
\[
\begin{bmatrix}
	A_x \\ A_y \\ A_z
\end{bmatrix}
=
\begin{bmatrix}
	\sin\theta\cos\phi & \cos\theta\cos\phi & -\sin\phi \\
	\sin\theta\sin\phi & \cos\theta\sin\phi & \cos\phi \\
	\cos\theta & -\sin\theta  & 0 \\
\end{bmatrix}
\begin{bmatrix}
	A_r \\ A_\theta \\ A_\phi
\end{bmatrix}
\]
\[
\begin{bmatrix}
	A_r \\ A_\theta \\ A_\phi
\end{bmatrix}
=
\begin{bmatrix}
	\sin\theta\cos\phi & \sin\theta\sin\phi & \cos\theta\\
	\cos\theta\cos\phi & \cos\theta\sin\phi & -\sin\theta\\
	-\sin\phi & \cos\phi  & 0 \\
\end{bmatrix}
\begin{bmatrix}
	A_x \\ A_y \\ A_z
\end{bmatrix}
\]

\extraline
Gradient
\[ \nabla V = \frac{\partial V}{\partial r}\uvec{r} + \frac{1}{r}\frac{\partial V}{\partial \theta}\uvec{\theta} + \frac{1}{r\sin\theta}\frac{\partial V}{\partial \phi}\uvec{\phi} \]

Laplacien
\[ \nabla^2 V = \frac{1}{r^2} \frac{\partial}{\partial r}\left( r^2 \frac{\partial V}{\partial r} \right)
+ \frac{1}{r^2\sin\theta}\frac{\partial}{\partial \theta} \left( \sin\theta \frac{\partial V}{\partial\theta} \right)
+ \frac{1}{r^2\sin^2\theta}\frac{\partial^2 V}{\partial \phi^2} \]

Éléments différentiels
\begin{gather*}
d\vec{l} = dr\,\uvec{r} + r\,d\theta\,\uvec{\theta} + r\sin\theta\,d\phi\,\uvec{\phi} \\
d\vec{s} = r^2\sin\theta\,d\theta\,d\phi\,\uvec{r} 
+ r\,dr\sin\theta\,d\phi\,\uvec{\theta}
+ r\,dr\,d\theta\,\uvec{\phi} \\
dv = r^2\,dr\,\sin\theta\,d\theta\,d\phi
\end{gather*}


\hrulefill
\section{Courant électrique}

Courant [$\si{\ampere} = \si{\coulomb}/\si{\sec}$]
\[ I \equiv \frac{dQ}{dt}
\qquad
I =\int \vec{J}_v \cdot d\vec{s} \qquad
I = \int \vec{J}_s \cdot d\vec{l}
\]

Densité de courant [$\si{\ampere/\metre^2}$]
\begin{gather*}
\vec{J}_v \equiv \rho_v \vec{v} = \sigma\vec{E} = \frac{1}{\rho} \vec{E} 
\end{gather*}
%
\begin{tabular}{@{\tableindent}ll@{}}
	Densité de charge volumique & $\rho_v$ \\
	Vitesse de dérive & $\vec{v}$\\
	Conductivité électrique & $\sigma$  \\
	Résistivité électrique  [$\si{\ohm\cdot\meter}$] & $\rho \equiv 1/\sigma$ \\
\end{tabular}

\extraline
Résistance [$\si{\ohm} = \si{\volt/\ampere}$]
\begin{gather*}
R \equiv \frac{L}{\sigma A} = \frac{V}{I}
= \frac{-\int \vec{E}\cdot{d\vec{l}}}{\int \sigma\vec{E}\cdot{d\vec{s}}}
= \int\frac{dl}{\sum_i \sigma_i A_i} > 0
\end{gather*}
\tableindent ($L$ doit être mesuré dans la même direction de $\vec{E}$) \\
\tableindent ($A$ doit être mesuré sur un plan $\perp$ à $\vec{E}$) \\
\halfline
\begin{tabular}{@{\tableindent}ll@{}}
	Résistances en série & $R_\text{tot} = R_1 + R_2 + \dots$ \\
	Résistances en parallèle & $ \frac{1}{R_\text{tot}} = \frac{1}{R_1} + \frac{1}{R_2} + \dots$\\
\end{tabular}

Équation de continuité et temps de relaxation [$\si{\sec}$]
\[ \nabla\cdot\vec{J}_v = -\frac{\partial\rho_v}{\partial t} 
\qquad
\rho_v(t) = \rho_v(0) \, e^{-t/\tau}
\qquad
\tau = \frac{\epsilon}{\sigma} \]

Loi de Joule : puissance [$\si{\watt} = \si{\joule}/\si{\sec}$]
\[ P 
\equiv \frac{dW}{dt}
= \int \vec{E}\cdot \vec{J}_v \,dv 
= \int \frac{1}{\sigma}\, J_v^2\,dv
= VI = I^2 R\]


\hrulefill
\section{Magnétostatique}

Loi de Biot-Savart : champ induction magnétique (``magnetic flux density") [$\si{\tesla} = \si{(\newton\cdot\sec)/(\coulomb\cdot\metre)}$]
\begin{align*}
\vec{B} &= \frac{\mu}{4\pi} \int_l \frac{I\,d\vec{l}\,' \times(\vec{r}-\vec{r}\,')}{|\vec{r}-\vec{r}\,'|^3}
 \qquad  \\
\vec{B} &= \frac{\mu}{4\pi} \int_s \frac{\vec{J}_s \times(\vec{r}-\vec{r}\,')}{|\vec{r}-\vec{r}\,'|^3} \, ds'
\qquad (\vec{J}_s: \si{\ampere/\metre}) \\ 
\vec{B} &= \frac{\mu}{4\pi} \int_v \frac{\vec{J}_v \times(\vec{r}-\vec{r}\,')}{|\vec{r}-\vec{r}\,'|^3} \, dv'
\qquad (\vec{J}_v: \si{\ampere/\metre^2})
\end{align*}

Champ magnétique [$\si{\ampere/\meter}$]
\[ \vec{H} = \frac{\vec{B}}{\mu} \]

Loi d'Ampère
\[ \oint_C \vec{H}\cdot d\vec{l} = I_\text{encerclé}
\qquad \nabla\times\vec{H} = \vec{J}_v   \]

Solénoïde infini
\[ \vec{B} = \mu n I  \qquad \text{(intérieur)} \qquad
\vec{B} = 0  \qquad \text{(extérieur)} \]
\[ |\vec{J}_s | = nI \]
\tableindent ($n$: nombre de tours par longueur) \\

\extraline
Force magnétique [\si{\newton}]
\[ \vec{F}_m = q\vec{v} \times \vec{B} 
\qquad
\vec{F}_m = \int I\,d\vec{l} \times \vec{B}_\text{ext}  \]

Flux magnétique
\[ \Phi = \int \vec{B}\cdot d\vec{s}\]


Théorème de Gauss pour champs magnétiques
\[ \oint \vec{B}\cdot d\vec{s} = 0 \qquad
\nabla\cdot \vec{B} = 0 \]

Moment de force et dipôles magnétiques
\[
\vec{\tau} = \vec{r}\times\vec{F} \qquad 
\vec{\tau} = \vec{m}\times\vec{B} \qquad
\textstyle 
\vec{m} = I\int d\vec{s} = IA\,\uvec{a}_n
\]

Magnétisation
\[ \vec{B} = \mu_0(\vec{H}+\vec{M}) = \mu\vec{H}\]
\[ \vec{M} = \chi_m \vec{H}
\qquad
\mu = \mu_0(1+\chi_m)
 \]

Conditions aux limites
\[ B_{n_1} = B_{n_2} \qquad B_{t_1} = H_{t_2} = J_s \]

Loi de Faraday
\[ \oint_C \vec{E} \cdot d\vec{l} = - \frac{d\Phi}{dt}
\qquad
\nabla\times\vec{E} = -\frac{\partial\vec{B}}{\partial t} \]

EMF et règle de flux
\[ \varepsilon = \int(\vec{v}\times\vec{B})\cdot d\vec{l} 
\qquad
\varepsilon = -N\frac{d\Phi}{dt} \]


\hrulefill
\section{Équations de Maxwell}
\begin{gather*}
\nabla \cdot \vec{D} = \rho_v \qquad
\nabla \cdot \vec{B} = 0 \\
\nabla \cdot \vec{E} = -\frac{\partial \vec{B}}{\partial t} \qquad
\nabla \times \vec{H} = \vec{J} + \frac{\partial \vec{D}}{\partial t}
\end{gather*}

\hrulefill
\section{Constantes}
Permittivité du vide
\[ 	\epsilon_0 = \SI{8,85E-12}{{\farad}/{\meter}} =  \SI{8,85E-12}{{\coulomb^2}/{(\newton\cdot\meter^2)}} \]
\[	k = {1}/{4\pi\epsilon_0} = \SI{9,00E9}{\newton\cdot\meter^2/\kilogram^2}\]

Perméabilité du vide
\[ \mu_0 = \SI{4\pi E-7}{H/\meter} = \SI{4\pi E-7}{\tesla\cdot\meter/\ampere} \]

Charge et masse de l'électron
\begin{gather*}
q_e = |e| = \SI{1,6E-19}{\coulomb}
\qquad m_e = \SI{9,11E-31}{\kilogram}
\end{gather*}

Vitesse de la lumière
\[ c = \SI{3E8}{\meter/\second}  \]

\hrulefill


\section{Rappel mathématique}

\subsection{Identités trigonométriques}
\halfline
\begin{multicols}{2}
\noindent
\[ \sin^2\theta +  \cos^2\theta = 1 \]
\begin{align*}
\cos 2\theta 
&=\cos^2\theta - \sin^2\theta \\
&=2\cos^2\theta - 1 \\
&=1  - 2\sin^2\theta 
\end{align*}
\[ \sin 2\theta  = 2\sin\theta\cos\theta  \]
\[ \sec^2\theta  = 1+ \tan^2\theta \]
\[ \tan 2\theta = \frac{2\tan\theta}{1-\tan^2\theta} \]
\[ \tan \theta = \pm\sqrt{\frac{1-\cos 2\theta}{1+\cos 2\theta}} = \frac{\sin\theta}{\cos\theta} \]
\end{multicols}
\[ \sin(A\pm B) = \sin A\cos B \pm \cos A\sin B\]
\[ \cos(A\pm B) = \cos A\cos B \mp \sin A\sin B\]

\subsection{Quelques dérivées}
\halfline
\begin{multicols}{2}
\noindent
\[ \deriv (ax^n) = nax^{n-1}\]
\[ \deriv (e^{ax}) = ae^{ax}\]
\[ \deriv (\ln ax) = \frac{a}{x} \]
\[ \deriv (\sin ax) = a\cos ax\]
\[ \deriv (\cos ax) = -a\sin ax\]
\[ \deriv (\tan ax) = a\sec^2 ax\]
\[ \deriv (\sec x) = \tan x \sec x \]
\[ \deriv (\cot x) = -a\csc^2 ax \]
\[ \deriv (\csc x) = -\cot x \csc x \]
\end{multicols}
\[ \deriv [f(x)g(x)] = f(x)g'(x) + g(x)f'(x)  \]
\[ \deriv \left[ \frac{f(x)}{g(x)} \right] = \frac{g(x)f'(x) - f(x)g'(x)}{[g(x)]^2}  \]


\subsection{Intégrales résolues}
\halfline
\begin{multicols}{2}
\noindent
\[ \int x^n \,dx =  \frac{x^{n+ 1}}{n+1} \quad (n\neq 1) \]
\[ \int e^{ax}\,dx = \frac{1}{a}e^{ax} \]
\[ \int \ln(ax) = x\ln(ax) - x \]
\[ \int \sin(ax)\,dx =-\frac{1}{a} \cos(ax) \]
\[ \int \cos(ax)\,dx =\frac{1}{a} \sin(ax) \]
\[ \int \sin^2(ax)\,dx = \frac{x}{2} - \frac{\sin(2ax)}{4a} \]
\[ \int \cos^2(ax)\,dx = \frac{x}{2} + \frac{\sin(2ax)}{4a} \]
%
\[ \int_a^b u\,dv = uv|_a^b - \int_a^b v\,du  \]
%
\[ \int \frac{dx}{x} = \ln \lvert x \rvert \]
\[ \int \frac{dx}{a+bx} = \frac{1}{b}\ln\lvert a+bx\rvert \]
\[ \int \frac{dx}{x^2+a^2} = \frac{1}{a}\arctan\left(\frac{x}{a}\right) \]
\[\int \frac{dx}{\sqrt{x^2+a^2}} = \ln \lvert x + \sqrt{x^2+a^2} \rvert \]
\[ \int \frac{dx}{(x^2+a^2)^{3/2}} = \frac{x}{a^2\sqrt{x^2+a^2}} \]
\[ \int \frac{dx}{(a+bx)^2} = -\frac{1}{b(a+bx)} \]
%
\end{multicols}

\extraline
Formes avec numérateur $x\,dx$
\[ \int \frac{x\,dx}{(x^2 \pm a^2)^{1/2}} = \sqrt{x^2 \pm a^2} \qquad
\int \frac{x\,dx}{(x^2+a^2)^{3/2}} = -\frac{1}{\sqrt{x^2+a^2}} \]
\[ \int \frac{x\,dx}{(x^2+a^2)^n} = -\frac{1}{2(n-1)} \frac{1}{(x^2+a^2)^{n-1}} \quad (n>0)  \]
\[ \int \frac{x\,dx}{(a^2-bx)^{3/2}} = \frac{2x}{b\sqrt{a^2-bx}} + \frac{4\sqrt{a^2-bx}}{b^2} \]

Formes avec numérateur $x^2\,dx$
\[ \int \frac{x^2\,dx}{\sqrt{x^2+a^2}} =  \frac{u}{2}\sqrt{x^2+a^2} - \frac{a^2}{2}\ln(x+\sqrt{x^2+a^2}) \]
%
\[ \int \frac{x^2\,dx}{(x^2+a^2)^{3/2}} = -\frac{x}{\sqrt{x^2+a^2}} + \ln \lvert x + \sqrt{x^2+a^2} \rvert \]


\subsection{Autres formules}
Lois des logarithmes
\begin{gather*}
\log_a (xy) = \log_a x + \log_a y \\
 \log_a (x/y) = \log_a x - \log_a y \\
\log_a (x^r) = r\log_a x
\end{gather*}

Géométrie \\
\halfline
\begin{tabular}{@{\tableindent}ll@{}}
	Cercle & $C = 2\pi r $, $A =\pi r^2$ \\
	Sphère & $A = 4\pi r^2$, $V = \frac{4}{3}\pi r^3 $ \\
\end{tabular}
\extraline

Résolution d'une équation quadratique
\[ ax^2 + bx + c = 0 \Longrightarrow x = \frac{-b \pm \sqrt{b^2-4ac} }{2a} \]

Théorèmes fondamentaux
\begin{gather*}
\int_{\vec{a}}^{\vec{b}} (\nabla f)\cdot d\vec{l} = f(\vec{b}) - f(\vec{a}) 
\qquad\text{(Théorème du gradient)} \\
\int (\nabla \cdot \vec{A}) \, dv = \oint \vec{A}\cdot d\vec{s}
\qquad\text{(Théorème de la divergence)} \\
\int (\nabla\times\vec{A})\cdot d\vec{s} = \oint \vec{A}\cdot d\vec{l}
\qquad\text{(Théorème du rotationnel)}
\end{gather*}






\hrulefill


\scriptsize

\href{https://github.com/zhouluke/PhysicsFormulas}{Feuille de formules}  \copyright\ 2021 Luke Zhou. \\
\href{http://wch.github.io/latexsheet/}{Modèle de document}  \copyright\ 2014 Winston Chang.


\end{multicols}
\end{document}
