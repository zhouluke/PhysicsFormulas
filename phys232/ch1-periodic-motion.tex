\chapter{Periodic Motion}
\section{Simple Harmonic Motion}
\begin{align*}
\shortintertext{Restoring forces for a displacement $x$:} 
F(x) &= -(k_1 + k_2x^2 + k_3x^3 + \cdots) 
\intertext{For small x, the upper order terms are negligible:} 
F(x) &= -k_1x 
\end{align*}
 
When a small mass is attached to the end of a massless spring:
\[ \boxed{-k_1x = m\frac{d^2x}{dt^2}} \] 

The solution to the differential equation gives a \textbf{sinusoidal} function: \[ \boxed{x = A \sin(\omega t+\varphi_0)} \] %
\begin{center} 
where \textbf{angular frequency} $\boxed{\omega = \sqrt{k_1/m}}$ \textit{(independent of A and $\varphi_0$)}

Period: $T=\frac{2\pi}{\omega} $; Frequency: $f = \frac{1}{T}=\frac{\omega}{2\pi} $   \\
\end{center}


Real vibrations don't go on forever; SHM models a \textbf{steady state of vibration}.

More conditions for using a sinusoidal SHM model:
\begin{itemize}
\item Small amplitudes
\item Simple systems
\item No damping
\item $F_{restoring} \propto x$ %
\end{itemize} 

\section{Rotating Vector Representation}
Represent SHM as the geometrical projection of the \textit{x-component} of uniform circular motion (use polar coordinates, where CCW is positive):
\begin{empheq}[left=\empheqlbrace]{align*}
\theta &= \omega t + \alpha \\
x&=A\cos\theta=A\cos(\omega t + \alpha)
\end{empheq}

Trig identity to convert from cos to sin: \[ \cos\theta=\sin\left(\theta + \frac{\pi}{2}\right) \]
%\[ \leadsto \varphi_0 = \alpha + \frac{\pi}{2} \]
%
\subsection{Extra axis}
\begin{empheq}[right=\empheqrbrace]{align*}
\text{Only \textit{x} is the \textit{measureable} component: } x &= r\cos\theta \\
\text{The y-component is \textit{ficticious}: } y &= r\sin\theta
\end{empheq}

\subsection{Rotations with $j$} 
\begin{align*} %
\mathbf{r}&=\mathbf{i}x+\mathbf{j}y  \\
\text{Rewrite as } \mathbf{r}&=x+\mathbf{j}y 
\end{align*}
Reinterpret $\mathbf{j}y$ as ``move a dist of \textit{y} along the y-axis"

Remove vector notation: \[ \boxed{z=x+jy} \]
Reinterpret $\mathit{j}$ as \textit{``90$^\circ$ rotation CCW from x-axis"} 

So, $\mathit{j^2}$ represents two 90$^\circ$ rotations CCW from x-axis.
\[ \boxed{j^2=-1; j=\sqrt{-1}} \]

Thus, $z=x+jy$ can be considered:
\begin{itemize}
\item Geometrically: as a vector of length $x=\sqrt{x^2+y^2}$ at an angle $\theta = \arctan(\frac{y}{x})$ to the x-axis
\item As a complex \# (with $j$)
\end{itemize}

\section{The Complex Exponential}
General Taylor series: $f(x)=\sum_{n=0}^\infty \frac{x^n}{n!} f^{(n)}(0)$ 

Use Taylor series for sin \& cos to obtain: 
\[ \cos\theta + j\sin\theta = 1+j\theta+\frac{(j\theta)^2}{2!}+\frac{(j\theta)^3}{3!}+\dots+\frac{(j\theta)^n}{n!}+\dots \]

Then, using the Taylor series for $e^{j\theta}$, we obtain a result known as \textbf{Euler's identity}:
\begin{equation}
	\boxed{\cos\theta + j\sin\theta = e^{j\theta}} \label{ch1:eq-eulers-identity}
\end{equation}

To \textit{rotate} the vector rep'd by complex \# $z$ an angle of $+\theta$, multiply by $\boxed{e^{j\theta}}$.

\subsection{Derivatives}
When analyzing periodic displacements, sometimes we get a differential equation of motion with terms involving the velocity \&/or acceleration.

Using the trigonometric functions:
\begin{empheq}[right=\empheqrbrace]{align*}
x&=A \cos(\omega t+\alpha) \\
\frac{dx}{dt}&=-\omega A \sin(\omega t+\alpha) \\
\frac{d^2x}{dt^2}&=-\omega^2 A \cos(\omega t+\alpha) 
\end{empheq}
Leads to an awkward mix of sin \& cos terms when the derivatives are subbed into the differential equation.

Using the complex exponential:
\begin{empheq}[right=\empheqrbrace]{align*}
z=A \cos(\omega t+\alpha)+jA \sin(\omega t+\alpha) &= Ae^{j(\omega t+\alpha)} \\
\frac{dz}{dt}=j\omega Ae^{j(\omega t+\alpha)} &= j\omega z \\
\frac{d^2z}{dt^2}=(j\omega)^2 Ae^{j(\omega t+\alpha)} &= -\omega^2 z 
\end{empheq}

Each $\frac{d}{dt}$ creates a phase shift of $+\pi/2$ (graphically, the vector undergoes a rotation). 

Physically meaningful projections of the complex exponential's derivatives: 
\[ x=\Re(z)  \hhtab \frac{dx}{dt}=\Re(j\omega z) \hhtab \frac{d^2x}{dt^2} = \Re(-\omega^2 z)\]



\section{Small angle approximations}
When $\theta$ is small and measured in radians,
\begin{align}
	\sin\theta &\approx \theta \label{ch1:eq-small-angle-sin} \\
	\tan\theta &\approx \theta  \label{ch1:eq-small-angle-tan} \\
	\cos\theta &\approx 1-\frac{\theta^2}{2} \label{ch1:eq-small-angle-cos}
\end{align}