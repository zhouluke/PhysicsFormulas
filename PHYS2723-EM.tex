\documentclass[10pt,landscape]{article}
\usepackage{multicol}
\usepackage{calc}
\usepackage{ifthen}
\usepackage[landscape]{geometry}
\usepackage{hyperref}
\usepackage{amsmath}

% To make this come out properly in landscape mode, do one of the following
% 1.
%  pdflatex latexsheet.tex
%
% 2.
%  latex latexsheet.tex
%  dvips -P pdf  -t landscape latexsheet.dvi
%  ps2pdf latexsheet.ps



% This sets page margins to .5 inch if using letter paper, and to 1cm
% if using A4 paper. (This probably isn't strictly necessary.)
% If using another size paper, use default 1cm margins.
\ifthenelse{\lengthtest { \paperwidth = 11in}}
	{ \geometry{top=.5in,left=.5in,right=.5in,bottom=.5in} }
	{\ifthenelse{ \lengthtest{ \paperwidth = 297mm}}
		{\geometry{top=1cm,left=1cm,right=1cm,bottom=1cm} }
		{\geometry{top=1cm,left=1cm,right=1cm,bottom=1cm} }
	}

% Turn off header and footer
\pagestyle{empty}
 

% Redefine section commands to use less space
\makeatletter
\renewcommand{\section}{\@startsection{section}{1}{0mm}%
                                {-1ex plus -.5ex minus -.2ex}%
                                {0.5ex plus .2ex}%x
                                {\normalfont\large\bfseries}}
\renewcommand{\subsection}{\@startsection{subsection}{2}{0mm}%
                                {-1explus -.5ex minus -.2ex}%
                                {0.5ex plus .2ex}%
                                {\normalfont\normalsize\bfseries}}
\renewcommand{\subsubsection}{\@startsection{subsubsection}{3}{0mm}%
                                {-1ex plus -.5ex minus -.2ex}%
                                {1ex plus .2ex}%
                                {\normalfont\small\bfseries}}
\makeatother

% Don't print section numbers
\setcounter{secnumdepth}{0}


\setlength{\parindent}{0pt}
\setlength{\parskip}{0pt plus 0.5ex}

\newcommand{\extraline}{\vspace{1em}}
% -----------------------------------------------------------------------

\begin{document}

\raggedright
\footnotesize
\begin{multicols}{3}


% multicol parameters
% These lengths are set only within the two main columns
%\setlength{\columnseprule}{0.25pt}
\setlength{\premulticols}{1pt}
\setlength{\postmulticols}{1pt}
\setlength{\multicolsep}{1pt}
\setlength{\columnsep}{2pt}

\begin{center}
     \Large{\textbf{Électricité et magnétisme}} \\
     \small{Luke Zhou $\cdot$ PHYS 2723 $\cdot$ Hiver 2021}
\end{center}

\section{Électrostatique}

Loi de Coulomb  
 \[ \vec{F} = q\vec{E}(r) = \frac{1}{4\pi\epsilon} \frac{kq_1q_2}{|r|^2} \hat{r}  \]

Charge  [C] ($'$: point de source; sans $'$: point d'observation)
\[  Q = \int_{l'} \rho_l(\vec{r}') \,dl'  \qquad  Q = \int_{s'} \rho_s(\vec{r}') \,ds'  \qquad Q = \int_{v'} \rho_v(\vec{r}') \,dv'  \]

Champ électrique [N/C, V/m]
\[ \vec{E}(\vec{r}) = \frac{1}{4\pi\epsilon} \int_{l'} \frac{\rho_l(\vec{r}')}{| \vec{r} - \vec{r}' |^3 } (\vec{r} - \vec{r}' )\,dl'  \]
%
\[\vec{E}(\vec{r}) = \frac{1}{4\pi\epsilon} \int_{s'} \frac{\rho_s(\vec{r}')}{| \vec{r} - \vec{r}'|^3 } (\vec{r} - \vec{r}' )\,ds' \]
%
\[\vec{E}(\vec{r}) = \frac{1}{4\pi\epsilon} \int_{v'} \frac{\rho_v(\vec{r}')}{| \vec{r} - \vec{r}' |^3 } (\vec{r} - \vec{r}' )\,dv'  \]

Densité de flux électrique (Champ $\vec{D}$)  [N$\cdot$F/C$\cdot$m, V$\cdot$F/m${}^2$]
\[ \vec{D} = \epsilon\vec{E}
\qquad \nabla \cdot \vec{D} = \rho_v \]

Flux électrique [C/m${}^2$]
\[ \Psi = \int_{S} \vec{D} \cdot d\vec{s} = \epsilon \int_{S} \vec{E} \cdot d\vec{s} \]

Théorème de Gauss
\[ \oint_S \vec{E}\cdot d\vec{s} = \frac{Q_\text{int}}{\epsilon}
\qquad  \oint_S \vec{D}\cdot d\vec{s} = Q_\text{int} \]

Potentiel [V = N$\cdot$m/C]
\[ {V}(\vec{r}) = \frac{1}{4\pi\epsilon} \int_{l'} \frac{\rho_l(\vec{r}')}{\| \vec{r} - \vec{r}' \| } \,dl'
\qquad  {V}(\vec{r}) = \frac{1}{4\pi\epsilon} \int_{s'} \frac{\rho_s(\vec{r}')}{\| \vec{r} - \vec{r}' \| } \,ds'  \]
%
\[ {V}(\vec{r}) = \frac{1}{4\pi\epsilon} \int_{v'} \frac{\rho_v(\vec{r}')}{\| \vec{r} - \vec{r}' \| } \,dv'  \]

Potentiel et champ électrique
$\Delta V = V_B - V_A  = -\int_A^B \vec{E}\cdot d\vec{l} \qquad \vec{E} = -\nabla V 
 \qquad \oint_C \vec{E}\cdot d\vec{l}  = 0 $

\extraline
$\vec{E}$ conservatif $\Leftrightarrow \nabla \times \vec{E} = 0$.
Électrostatique $\Rightarrow \vec{E}$ conservatif.\\ 

\extraline
Travail effectué par une force extérieure en déplaçant une charge
\[ W_{a\rightarrow b} = -q\int_{a}^{b} \vec{E}\cdot d\vec{l} = q\Delta V \]

Dipôle électrique
\[ V(\vec{r}) = \frac{q}{4\pi\epsilon_0} \frac{d\cos\theta}{r^2} = \frac{\vec{p}\cdot\hat{r}}{4\pi\epsilon|r|^2} \]

Moment dipolaire: pointe de $q-$ vers $q+$ ($\hat{p}$ est souvent $\hat{z}$)
\[ \vec{p} = qd\cdot \hat{p} \]

Conditions aux limites: conducteurs
\[ E_n = \frac{\rho_s}{\epsilon} \qquad E_t = 0 \qquad \vec{E} = 0 \qquad \rho_v = 0 \]

Conditions aux limites: diélectriques
\[ E_{1t} = E_{2t}  \]
\[ (\vec{D}_1 - \vec{D}_2) \cdot \hat{n} = \rho_s  \qquad (\rho_s=0 \Rightarrow \epsilon_1 E_{1n} = \epsilon_2 E_{2n}) \] 

\extraline
Propriétés des diélectriques
\[ \vec{D} = \epsilon_0\vec{E} + \vec{P} = \epsilon_0(1+\chi)\vec{E} = \epsilon_0\epsilon_r\vec{E} = \epsilon\vec{E}
\qquad   \]
\begin{tabular}{@{}ll@{}}
	Polarisation  & $\vec{P} = \epsilon_0\chi\vec{E}$ \\
	Susceptibilité diélectrique  & $\chi$ \\
	Permittivité relative &  $\epsilon_r=1+\chi$
\end{tabular}

\extraline
Propriétés des condensateurs \\
$E_\text{total} = \frac{\sigma}{\epsilon_0} \qquad E = -\frac{\Delta V}{d} 
\qquad C = \frac{Q}{\Delta V} = \frac{A\epsilon}{d}$ \\

\extraline
\begin{tabular}{@{}ll@{}}
	Condensateurs en parallèle  & $C_\text{total} = C_1 + C_2 + \dots$ \\
	Condensateurs en série & $\frac{1}{C_\text{total}} = \frac{1}{C_1} + \frac{1}{C_2} + \dots $
\end{tabular}

\extraline
\hrulefill



\section{Systèmes de coordonnées}

\subsection{Coordonnées cylindriques $(\rho, \phi, z)$}

Restrictions: $\qquad \phi\in[0,2\pi]$
\[ x = \rho\cos\phi \qquad y = \rho\sin\phi \qquad z=z  \]
\[ \rho = \sqrt{x^2 + y^2} \qquad \phi = \arctan\left(\frac{y}{x}\right) \]
%
\[
\begin{bmatrix}
	A_x \\ A_y \\ A_z
\end{bmatrix}
=
\begin{bmatrix}
	\cos\phi & -\sin\phi & 0 \\
	\sin\phi & \cos\phi & 0 \\
	0 & 0 & 1 \\
\end{bmatrix}
\begin{bmatrix}
	A_\rho \\ A_\phi \\ A_z
\end{bmatrix}
\]
\[
\begin{bmatrix}
	A_\rho \\ A_\phi \\ A_z
\end{bmatrix}
=
\begin{bmatrix}
	\cos\phi & \sin\phi & 0 \\
	-\sin\phi & \cos\phi & 0 \\
	0 & 0 & 1 \\
\end{bmatrix}
\begin{bmatrix}
	A_x \\ A_y \\ A_z
\end{bmatrix}
\]

\subsection{Coordonnées sphériques $(r, \theta, \phi)$}

Restrictions: $\qquad\theta\in[0,\pi] \qquad \phi\in[0,2\pi]$
%
\[ x = r\sin\theta\cos\phi \qquad y =  r\sin\theta\sin\phi \qquad z= r\cos\theta  \]
\[ r = \sqrt{x^2 + y^2 + z^2} \qquad \theta = \arccos\left(\frac{z}{r}\right) \qquad \phi = \arctan\left(\frac{y}{x}\right) \]
%
\[
\begin{bmatrix}
	A_x \\ A_y \\ A_z
\end{bmatrix}
=
\begin{bmatrix}
	\sin\theta\cos\phi & \cos\theta\cos\phi & -\sin\phi \\
	\sin\theta\sin\phi & \cos\theta\sin\phi & \cos\phi \\
	\cos\theta & -\sin\theta  & 0 \\
\end{bmatrix}
\begin{bmatrix}
	A_r \\ A_\theta \\ A_\phi
\end{bmatrix}
\]
\[
\begin{bmatrix}
	A_r \\ A_\theta \\ A_\phi
\end{bmatrix}
=
\begin{bmatrix}
	\sin\theta\cos\phi & \sin\theta\sin\phi & \cos\theta\\
	\cos\theta\cos\phi & \cos\theta\sin\phi & -\sin\theta\\
	-\sin\phi & \cos\phi  & 0 \\
\end{bmatrix}
\begin{bmatrix}
	A_x \\ A_y \\ A_z
\end{bmatrix}
\]

\hrulefill

\section{Rappel de calcul}

\subsection{Intégrales résolus}
\begin{multicols}{2}
\[ \int x^n \,dx =  \frac{x^{n+ 1}}{n+1} \quad (n\neq 1) \]
\[ \int e^{ax}\,dx = \frac{1}{a}e^{ax} \]
\[ \int \frac{dx}{x} = \ln \lvert x \rvert \]
\[ \int \ln(ax) = x\ln(ax) - x \]
\[ \int \frac{dx}{a+bx} = \frac{1}{b}\ln\lvert a+bx\rvert \]
\[ \int \sin(ax)\,dx =-\frac{1}{a} \cos(ax) \]
\[ \int \cos(ax)\,dx =\frac{1}{a} \sin(ax) \]
\[ \int \sin^2(ax)\,dx = \frac{x}{2} - \frac{\sin(2ax)}{4a} \]
\[ \int \cos^2(ax)\,dx = \frac{x}{2} + \frac{\sin(2ax)}{4a} \]
%
\[ \int \frac{dx}{x^2+a^2} = \frac{1}{a}\arctan\left(\frac{x}{a}\right) \]
\[\int \frac{dx}{\sqrt{x^2+a^2}} = \ln \lvert x + \sqrt{x^2+a^2} \rvert \]
\[ \int \frac{dx}{(x^2+a^2)^{3/2}} = \frac{x}{a^2\sqrt{x^2+a^2}} \]
\[ \int \frac{x\,dx}{(x^2+a^2)^{3/2}} = -\frac{1}{\sqrt{x^2+a^2}} \]
\[ \int \frac{dx}{(a+bx)^2} = -\frac{1}{b(a+bx)} \]
\end{multicols}
\[ \int \frac{x^2\,dx}{(x^2+a^2)^{3/2}} = -\frac{x}{\sqrt{x^2+a^2}} + \ln \lvert x + \sqrt{x^2+a^2} \rvert\]
\[ \int \frac{x\,dx}{(a^2-bx)^{3/2}} = \frac{2x}{b\sqrt{a^2-bx}} + \frac{4\sqrt{a^2-bu}}{b^2} \]

\extraline
\hrulefill



\section{Constantes}
\[ \epsilon_0 = 8,85\times10^{-12} \frac{\text{F}}{\text{m}} =  8,85\times10^{-12} \frac{\text{C}^2}{\text{N$\cdot$m${}^2$}} \]
\[ q_e = |e| = 1,6 \times 10^{-19} \text{ C}
\qquad m_e = 9,11 \times 10^{-31} \text{ kg} \]
\[ \mu_0 = 4\pi \times 10^{-7} \text{ H/m}
\qquad c = 3 \times 10^{8} \text{ m/s} \]

\hrulefill
\scriptsize

\href{https://github.com/zhouluke/PhysicsFormulas}{Cheatsheet}  \copyright\ 2021 Luke Zhou \\
\href{http://wch.github.io/latexsheet/}{Template}  \copyright\ 2014 Winston Chang


\end{multicols}
\end{document}
