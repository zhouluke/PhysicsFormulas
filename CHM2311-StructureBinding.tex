% !TeX spellcheck = en_US
\documentclass[10pt,landscape]{article}
\usepackage{multicol}
\usepackage{calc}
\usepackage{ifthen}
\usepackage[landscape]{geometry}
\usepackage{hyperref}
\usepackage{amsmath, amsfonts}
\usepackage{mathtools}
%\usepackage[h]{esvect}

\usepackage{siunitx}
\usepackage{physics}

% English spacing
\usepackage{icomma}
\frenchspacing
\usepackage[english]{babel}
 \sisetup{output-decimal-marker = {.}}
 
 
% To make this come out properly in landscape mode, do one of the following
% 1.
%  pdflatex latexsheet.tex
%
% 2.
%  latex latexsheet.tex
%  dvips -P pdf  -t landscape latexsheet.dvi
%  ps2pdf latexsheet.ps



% This sets page margins to .5 inch if using letter paper, and to 1cm
% if using A4 paper. (This probably isn't strictly necessary.)
% If using another size paper, use default 1cm margins.
\ifthenelse{\lengthtest { \paperwidth = 11in}}
	{ \geometry{top=.5in,left=.5in,right=.5in,bottom=.5in} }
	{\ifthenelse{ \lengthtest{ \paperwidth = 297mm}}
		{\geometry{top=1cm,left=1cm,right=1cm,bottom=1cm} }
		{\geometry{top=1cm,left=1cm,right=1cm,bottom=1cm} }
	}

% Turn off header and footer
\pagestyle{empty}
 

% Redefine section commands to use less space
\makeatletter
\renewcommand{\section}{\@startsection{section}{1}{0mm}%
                                {-1ex plus -.5ex minus -.2ex}%
                                {0.5ex plus .2ex}%x
                                {\normalfont\large\bfseries}}
\renewcommand{\subsection}{\@startsection{subsection}{2}{0mm}%
                                {-1explus -.5ex minus -.2ex}%
                                {0.5ex plus .2ex}%
                                {\normalfont\normalsize\bfseries}}
\renewcommand{\subsubsection}{\@startsection{subsubsection}{3}{0mm}%
                                {-1ex plus -.5ex minus -.2ex}%
                                {1ex plus .2ex}%
                                {\normalfont\small\bfseries}}
\makeatother

% Don't print section numbers
\setcounter{secnumdepth}{0}


\setlength{\parindent}{0pt}
\setlength{\parskip}{0pt plus 0.5ex}

\newcommand{\extraline}{\vspace{1em}}
\newcommand{\halfline}{\vspace{0.5em}}
\newcommand{\deriv}{\ensuremath{\frac{d}{dx}}}
\newcommand{\tableindent}{\hspace{1.5em}}
\newcommand{\uvec}[1]{\ensuremath{{\hat{#1}}}}
% -----------------------------------------------------------------------

\begin{document}

\raggedright
\footnotesize
\begin{multicols}{3}

% multicol parameters
% These lengths are set only within the two main columns
%\setlength{\columnseprule}{0.25pt}
\setlength{\premulticols}{1pt}
\setlength{\postmulticols}{1pt}
\setlength{\multicolsep}{1pt}
\setlength{\columnsep}{2pt}

\begin{center}
     \Large{\textbf{Structure \& Bonding}} \\
     \small{Luke Zhou $\cdot$ CHM 2311 $\cdot$ Winter 2022}
\end{center}

\section{Chapter 2}

Wavelength, frequency and the speed of light
\[ c = \lambda\nu  = \SI{3E8}{\meter/\second}  \]

Energy quantization
\[E = \frac{hc}{\lambda} = h\nu  \qquad
 h = \SI{6.626E-34}{\joule\cdot\second}  \]

Energy emitted by a hydrogen atom
 \[ \Delta E = E_{h} - E_{l} = R_H \left(  \frac{1}{n_l^2} - \frac{1}{n_h^2} \right) \] 
 \[ R_H =  \SI{2.179E-18}{\joule} \]

\extraline
Hydrogen emission spectra series \\
\begin{tabular}{@{\tableindent}ll@{}}
	Lyman series & $n_l = 1$ \\
	Balmer series & $n_l = 2$ \\
	Paschen series & $n_l = 3$ \\
	Brackett series & $n_l = 4$ \\
	Pfund series & $n_l = 5$ \\
	Humphreys series & $n_l = 6$ \\
\end{tabular}
\extraline

Wave-Particle Duality  
\[  mv = \frac{h}{\lambda} \]


Heisenberg Uncertainty Principle
\[  \Delta x \Delta p \geq \frac{h}{4\pi} \]

Schrödinger Wave Equation
\[ H\Psi = E\Psi  \]
\[ \left[ \frac{-h^2}{8\pi^2m} +
V(x,y,z)\right] \Psi(x,y,z) = E\Psi(x,y,z)  \]
	
Particle in a box
\[ \Psi(x) = \sqrt{\frac{2}{a}} \sin\left( \frac{n\pi x}{a} \right)  \qquad n \in \mathbb{N} \]
\[ E = \frac{n^2h^2}{8ma^2} \]

Particle on a ring
\[ \Psi(x) = B \exp \left(\frac{in\pi x}{L} \right)
\qquad n \in \mathbb{Z}, L = n\lambda  \]
\[ E = \frac{h^2}{8mL^2} (2n)^2 \]

Hydrogenic wave functions
\[ \Psi(r,\theta,\phi) = R(r) Y(\theta,\phi) \]

Quantum numbers \\
\extraline
\begin{tabular}{@{\tableindent}lp{1.5cm}lp{3cm}@{}}
$n$ & Principal & $1, 2, 3, 4, \ldots $ & orbital size \\
$l$ & Angular momentum & $0, 1, 2 \dots  n-1 $ & orbital shape \\
$m_l$ & Magnetic & $0, \pm 1, \pm 2, \ldots \pm l $ & orbital orientation \\
$m_s$ & Spin & $-\frac{1}{2}, +\frac{1}{2} $ & 4th quantum number \\
\end{tabular}

\tableindent $l$ can also be described with the letters $s, p, d, f, g, \ldots$

Radial \& angular nodes

\begin{tabular}{@{\tableindent}lll@{}}
\# of radial nodes & $n-l-1$ & $R(r) = 0$ (spheres)\\
\# of angular nodes & $l$ &  $Y(\theta,\phi)=0$ (planes \& cones)\\
Total \# of nodes & $n-1$ & &
\end{tabular}



\extraline
\hrulefill







\hrulefill


\scriptsize

\href{https://github.com/zhouluke/PhysicsFormulas}{Typography}  \copyright\ 2022 Luke Zhou. \\
Based on the course CHM 2311 Winter 2022 at the University of Ottawa \\
\href{http://wch.github.io/latexsheet/}{Document template}  \copyright\ 2014 Winston Chang.


\end{multicols}
\end{document}
