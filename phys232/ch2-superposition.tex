\chapter{Superposition of Periodic Motions}
The resultant of 2+ harmonic oscillators is the sum of the individual vibrations, if the system is \textbf{linear} (that is, if $F_{restoring}\propto x$).

\section{In 1D}

\subsection{Two Vibrations of Same Frequency}
Consider:
\begin{empheq}[left=\empheqlbrace]{align*}
OP_1: x_1&=A_1\cos(\omega t + \alpha_1) \\
OP_2: x_2&=A_2\cos(\omega t + \alpha_2)
\end{empheq}
Combined vibration: \[ OP: x = A\cos(\omega t+\alpha) \]

All 3 vectors $OP_1, OP_2$ \& $OP$ rotate at the same frequency. 

Let $\beta = \angle P_1OP$.

Phase constant of the combined vib'n: $\alpha = \alpha_1 + \beta$

\subsubsection{Using complex exponentials}
\begin{align*}
z = z_1 + z_2 &= A_1 e^{j(\omega t+\alpha_1)} + A_2 e^{j(\omega t+\alpha_2)} \\
&=e^{j(\omega t+\alpha_1)}[A_1 + A_2e^{j(\alpha_2 - \alpha_1)}]
\end{align*} %
%
\subsubsection{Special Case: Equal Amplitudes} 

Let \textbf{phase difference} $\delta = \alpha_2-\alpha_1$ \\
From geometry: $ A=2A_1\cos\beta = 2A_1\cos(\delta/2)$

Application: alternating maxima/minimae superposition pattern from two sources converging at a far away point, at any point on line $OB$ (as shown).

\subsection{Two Vibrations of Different Frequencies}
Let the initial phase shifts of both vibrations be zero.
Consider:
\begin{empheq}[left=\empheqlbrace]{align*}
x_1&=A_1\cos(\omega_1 t) \\ 
x_2&=A_2\cos(\omega_2 t)
\end{empheq}

Combined vibration displacement $OX$: \[ 0 \leq OX \leq A_1+A_2 \]

The combined motion will only be periodic if the periods of the component vibrations are \textbf{commensurable} - i.e. $\exists  (n_1$ \& $n_2)\in \mathbb{Z} $ such that $ T = n_1 T_1 = n_2 T_2 $ (use the smallest values of $n_1$ \& $n_2$ possible). 

\subsubsection{Beats}  
If the two frequencies are close in frequency: \textbf{beats} will be produced.

Combined vibration will be a disturbance with $\omega=\frac{\omega_1+\omega_2}{2}$, but with an \textit{amplitude that varies periodically with time}.

Add the $x_1$ \& $x_2$ from above. Use the trig identities for $\cos(\theta+\varphi)$ and $\cos(\theta-\varphi)$ to yield:
\[ x = 2A\cos\left(\frac{\omega_1-\omega_2}{2} t\right)\cos\left(\frac{\omega_2+\omega_1}{2} t\right) \]
This is only physically meaningful if $|\omega_1-\omega_2| \ll |\omega_1+\omega_2|$.

Combined displacement can be fitted within an envelope:
\[ x = \pm 2A\cos\left(\frac{\omega_1-\omega_2}{2} t\right) \]

Node-to-node (see diagram): this is half a cycle (a time equal to $\frac{2\pi}{|\omega_1-\omega_2|}$.) \\
So, the aurally observed \textbf{beat frequency} is $|\omega_1-\omega_2|$.

\subsection{Many Vibrations of Same Frequency \& Amplitude}
Let the component vibrations have equal, successive phase differences ($\delta$):
\[ x = A_0\cos(\omega t) \]
Resultant: \[ X = A\cos(\omega t + \alpha) \]

The combining vectors form a polygon that can be inscribed inside a circle.
Each $A_0$ subtends an angle of $\delta$. The resultant vector $A$ subtends an angle $N\delta$.
So:
\begin{empheq}[left=\empheqlbrace]{align*}
A &= N\delta=2R\sin(N\delta/2) \\
A_0 &= 2R\sin(\delta/2) 
\end{empheq}

Solve to get: 
\[ A =A_0 \frac{\sin(N\delta/2)}{\sin(\delta/2)} \] 

\begin{align*}
\intertext{Phase angle $\alpha$ of the combined oscillation: angle between resultant vector \& the 1st vector}
\alpha &= \angle COB - \angle COP \\
&= (90\degr - \delta/2)-(90\degr - N\delta/2) \tag{By properties of isosceles triangles} \\
&= \frac{(N-1)\delta}{2}
\end{align*}

Putting it all together:
\[ \boxed{X = A_0 \frac{\sin(N\delta/2)}{\sin(\delta/2)} \cos\left[\omega t + \frac{(N-1)\delta}{2}\right]} \]

\section{In 2D: Two Vibrations at $\perp$ Angles}
The resultant motion has \textit{two} real components - $x$ and $y$:
\begin{empheq}[left=\empheqlbrace]{align*}
x&=A_1 \cos(\omega t) \\
y&=A_2 \cos(\omega t + \delta)
\end{empheq}

Depending on the value of $\delta$, we'll get different path patterns described by these \textit{parametric equations}. (Be mindful of the direction of tracing - CCW or CW.)

(Study the analytical/mathematical and semicircle-sketching methods to determine the pattern)

\subsubsection{Different Frequencies: Lissajous Figures} 
The patterns depend on the $\omega_1:\omega_2$ ratio.
If $\omega_1$ and $\omega_2$ are not \textit{commensuable}, a periodically repeating pattern will not be produced.