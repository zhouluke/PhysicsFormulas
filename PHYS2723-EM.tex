% !TeX spellcheck = fr_FR
\documentclass[10pt,landscape]{article}
\usepackage{multicol}
\usepackage{calc}
\usepackage{ifthen}
\usepackage[landscape]{geometry}
\usepackage{hyperref}
\usepackage{amsmath}
\usepackage[h]{esvect}

\usepackage{siunitx}
\usepackage{physics}

% French spacing
\usepackage{icomma}
\frenchspacing
\usepackage[french]{babel}
 \sisetup{output-decimal-marker = {,}}
 
 
% To make this come out properly in landscape mode, do one of the following
% 1.
%  pdflatex latexsheet.tex
%
% 2.
%  latex latexsheet.tex
%  dvips -P pdf  -t landscape latexsheet.dvi
%  ps2pdf latexsheet.ps



% This sets page margins to .5 inch if using letter paper, and to 1cm
% if using A4 paper. (This probably isn't strictly necessary.)
% If using another size paper, use default 1cm margins.
\ifthenelse{\lengthtest { \paperwidth = 11in}}
	{ \geometry{top=.5in,left=.5in,right=.5in,bottom=.5in} }
	{\ifthenelse{ \lengthtest{ \paperwidth = 297mm}}
		{\geometry{top=1cm,left=1cm,right=1cm,bottom=1cm} }
		{\geometry{top=1cm,left=1cm,right=1cm,bottom=1cm} }
	}

% Turn off header and footer
\pagestyle{empty}
 

% Redefine section commands to use less space
\makeatletter
\renewcommand{\section}{\@startsection{section}{1}{0mm}%
                                {-1ex plus -.5ex minus -.2ex}%
                                {0.5ex plus .2ex}%x
                                {\normalfont\large\bfseries}}
\renewcommand{\subsection}{\@startsection{subsection}{2}{0mm}%
                                {-1explus -.5ex minus -.2ex}%
                                {0.5ex plus .2ex}%
                                {\normalfont\normalsize\bfseries}}
\renewcommand{\subsubsection}{\@startsection{subsubsection}{3}{0mm}%
                                {-1ex plus -.5ex minus -.2ex}%
                                {1ex plus .2ex}%
                                {\normalfont\small\bfseries}}
\makeatother

% Don't print section numbers
\setcounter{secnumdepth}{0}


\setlength{\parindent}{0pt}
\setlength{\parskip}{0pt plus 0.5ex}

\newcommand{\extraline}{\vspace{1em}}
\newcommand{\halfline}{\vspace{0.5em}}
\newcommand{\deriv}{\ensuremath{\derivative{}{x}{}}}
% -----------------------------------------------------------------------

\begin{document}

\raggedright
\footnotesize
\begin{multicols}{3}


% multicol parameters
% These lengths are set only within the two main columns
%\setlength{\columnseprule}{0.25pt}
\setlength{\premulticols}{1pt}
\setlength{\postmulticols}{1pt}
\setlength{\multicolsep}{1pt}
\setlength{\columnsep}{2pt}

\begin{center}
     \Large{\textbf{Électricité et magnétisme}} \\
     \small{Luke Zhou $\cdot$ PHYS 2723 $\cdot$ Hiver 2021}
\end{center}

\section{Électrostatique}

Loi de Coulomb : force électrique [N]
 \[ \vv{F} = q\vv{E}(r) \] %= \frac{1}{4\pi\epsilon} \frac{kq_1q_2}{|r|^2} \hat{r}  \]

Charge  [C] ($'$: point de source; sans $'$: point d'observation)
\[  Q = \int_{l} \rho_l(\vv{r}') \,dl'  \qquad  Q = \int_{s} \rho_s(\vv{r}') \,ds'  \qquad Q = \int_{v} \rho_v(\vv{r}') \,dv'  \]

Champ électrique [N/C, V/m]
\[ \vv{E}(\vv{r}) = \frac{1}{4\pi\epsilon} \int_{l} \frac{\rho_l(\vv{r}')}{| \vv{r} - \vv{r}' |^3 } (\vv{r} - \vv{r}' )\,dl'  \]
%
\[\vv{E}(\vv{r}) = \frac{1}{4\pi\epsilon} \int_{s} \frac{\rho_s(\vv{r}')}{| \vv{r} - \vv{r}'|^3 } (\vv{r} - \vv{r}' )\,ds' \]
%
\[\vv{E}(\vv{r}) = \frac{1}{4\pi\epsilon} \int_{v} \frac{\rho_v(\vv{r}')}{| \vv{r} - \vv{r}' |^3 } (\vv{r} - \vv{r}' )\,dv'  \]

Densité de flux électrique (Champ $\vv{D}$)  [C/m${}^2$]
\[ \vv{D} = \epsilon\vv{E}
\qquad \nabla \cdot \vv{D} = \rho_v \]

Flux électrique [C]
\[ \Psi = \int_{s} \vv{D} \cdot d\vv{s} = \epsilon \int_{s} \vv{E} \cdot d\vv{s} \]

Théorème de Gauss
\[ \oint_s \vv{E}\cdot d\vv{s} = \frac{Q_\text{int}}{\epsilon}
\qquad  \oint_s \vv{D}\cdot d\vv{s} = Q_\text{int} \]

Potentiel [V = N$\cdot$m/C]
\[ {V}(\vv{r}) = \frac{1}{4\pi\epsilon} \int_{l} \frac{\rho_l(\vv{r}')}{| \vv{r} - \vv{r}' | } \,dl'
\qquad  {V}(\vv{r}) = \frac{1}{4\pi\epsilon} \int_{s} \frac{\rho_s(\vv{r}')}{| \vv{r} - \vv{r}'| } \,ds'  \]
%
\[ {V}(\vv{r}) = \frac{1}{4\pi\epsilon} \int_{v} \frac{\rho_v(\vv{r}')}{| \vv{r} - \vv{r}' | } \,dv'  \]

Potentiel et champ électrique
\[ \Delta V = V_b - V_a  = -\int_a^b \vv{E}\cdot d\vv{l} \qquad \vv{E} = -\nabla V 
 \qquad \oint_C \vv{E}\cdot d\vv{l}  = 0 \]

$\vv{E}$ conservatif $\Leftrightarrow \nabla \times \vv{E} = 0$.
Électrostatique $\Rightarrow \vv{E}$ conservatif.\\ 

\extraline
Travail effectué par une force extérieure en déplaçant une charge
\[ W_{a\rightarrow b} = -q\int_{a}^{b} \vv{E}\cdot d\vv{l} = q\Delta V \]

Dipôle électrique
\[ V(\vv{r}) = \frac{q}{4\pi\epsilon_0} \frac{d\cos\theta}{r^2} = \frac{\vv{p}\cdot\vv{r}}{4\pi\epsilon|\vv{r}|^2} \]

Moment dipolaire [C$\cdot$m]
\[ \vv{p} = qd\cdot \hat{p} \qquad \text{($\hat{p}$ pointe de $q-$ vers $q+$;  $\hat{p}\parallel\hat{z}$ souvent)} \]

Conditions aux limites: conducteurs
\[ E_n = \frac{\rho_s}{\epsilon} \qquad E_t = 0 \qquad \vv{E}_\text{int} = 0 \qquad \rho_v = 0 \]

Conditions aux limites: diélectriques
\[ \vv{E}_{1t} =\vv{ E}_{2t}  \]
\[ ({D}_1 - {D}_2) \cdot \hat{n} = \rho_s  \qquad (\rho_s=0 \Rightarrow \epsilon_1 {E}_{1n} = \epsilon_2 {E}_{2n}) \] 

\extraline
Propriétés des diélectriques
\[ \vv{D} = \epsilon_0\vv{E} + \vv{P} = \epsilon_0(1+\chi)\vv{E} = \epsilon_0\epsilon_r\vv{E} = \epsilon\vv{E}
\qquad   \]
\begin{tabular}{@{}ll@{}}
	Polarisation  & $\vv{P} = \epsilon_0\chi\vv{E}$ \\
	Susceptibilité diélectrique  & $\chi$ \\
	Permittivité relative &  $\epsilon_r=1+\chi$
\end{tabular}

\extraline
Propriétés des condensateurs  \\
\extraline
\begin{tabular}{@{}ll@{}}
	Champ électrique & $ E = {\sigma}/{\epsilon_0}  = -{\Delta V}/{d} $ \\
	Capacitance [F] & $C \equiv  | {Q}/{\Delta V} | > 0 $ \\
	Condensateur à électrodes parallèles & $C = {A\epsilon}/{d} $ \\
	Condensateurs en parallèle  & $C_\text{total} = C_1 + C_2 + \dots$ \\
	Condensateurs en série & $\frac{1}{C_\text{total}} = \frac{1}{C_1} + \frac{1}{C_2} + \dots $
\end{tabular}

\extraline
\hrulefill



\section{Systèmes de coordonnées}

\subsection{Coordonnées cartésiennes $(x,y,z)$}

Gradient
\[ \nabla V = \frac{\partial V}{\partial x}\hat{x} + \frac{\partial V}{\partial y}\hat{y} + \frac{\partial V}{\partial z}\hat{z} \]

Laplacien
\[ \nabla^2 V = \frac{\partial^2 V}{\partial x^2} + \frac{\partial^2 V}{\partial y^2} + \frac{\partial^2 V}{\partial z^2} \]

Éléments différentiels
\[ \vv{dl} = dx\,\hat{x} + dy\,\hat{y} + dz\,\hat{z} \]
\[\vv{ds} = dy\,dz\,\hat{x} + dx\,dz\,\hat{y} + dx\,dy\,\hat{z} \]
\[dv = dx\,dy\,dz\]

\subsection{Coordonnées cylindriques $(\rho, \phi, z)$}

Restrictions: $\qquad \phi\in[0,2\pi]$
\[ x = \rho\cos\phi \qquad y = \rho\sin\phi \qquad z=z  \]
\[ \rho = \sqrt{x^2 + y^2} \qquad \phi = \arctan\left(\frac{y}{x}\right) \]
%
\[
\begin{bmatrix}
	A_x \\ A_y \\ A_z
\end{bmatrix}
=
\begin{bmatrix}
	\cos\phi & -\sin\phi & 0 \\
	\sin\phi & \cos\phi & 0 \\
	0 & 0 & 1 \\
\end{bmatrix}
\begin{bmatrix}
	A_\rho \\ A_\phi \\ A_z
\end{bmatrix}
\]
\[
\begin{bmatrix}
	A_\rho \\ A_\phi \\ A_z
\end{bmatrix}
=
\begin{bmatrix}
	\cos\phi & \sin\phi & 0 \\
	-\sin\phi & \cos\phi & 0 \\
	0 & 0 & 1 \\
\end{bmatrix}
\begin{bmatrix}
	A_x \\ A_y \\ A_z
\end{bmatrix}
\]

\extraline
Gradient
\[ \nabla V = \frac{\partial V}{\partial \rho}\hat{\rho} + \frac{1}{\rho}\frac{\partial V}{\partial \phi}\hat{\phi} + \frac{\partial V}{\partial z}\hat{z} \]

Laplacien
\[ \nabla^2 V = 
\frac{1}{\rho} \frac{\partial}{\partial \rho} \left( \rho\frac{\partial V}{\partial \rho} \right)
+ \frac{1}{\rho^2}\frac{\partial^2 V}{\partial \phi^2}
+ \frac{\partial^2 V}{\partial z^2} \]

Éléments différentiels
\[ \vv{dl} = d\rho\,\hat{\rho} + \rho\,d\phi\,\hat{\phi} + dz\,\hat{z} \]
\[\vv{ds} = \rho\,d\phi\,dz\,\hat{\rho} + d\rho\,dz\,\hat{\phi} + \rho\,d\rho\,d\phi\,\hat{z} \]
\[dv = \rho\,d\rho\,d\phi\,dz\]

\subsection{Coordonnées sphériques $(r, \theta, \phi)$}

Restrictions: $\qquad\theta\in[0,\pi] \qquad \phi\in[0,2\pi]$ \\
$\phi$ est l'angle formé dans le plan $xy$. \\
$\theta$ est l'angle entre le vecteur de position et l'axe des $z$.
%
\[ x = r\sin\theta\cos\phi \qquad y =  r\sin\theta\sin\phi \qquad z= r\cos\theta  \]
\[ r = \sqrt{x^2 + y^2 + z^2} \qquad \theta = \arccos\left(\frac{z}{r}\right) \qquad \phi = \arctan\left(\frac{y}{x}\right) \]
%
\[
\begin{bmatrix}
	A_x \\ A_y \\ A_z
\end{bmatrix}
=
\begin{bmatrix}
	\sin\theta\cos\phi & \cos\theta\cos\phi & -\sin\phi \\
	\sin\theta\sin\phi & \cos\theta\sin\phi & \cos\phi \\
	\cos\theta & -\sin\theta  & 0 \\
\end{bmatrix}
\begin{bmatrix}
	A_r \\ A_\theta \\ A_\phi
\end{bmatrix}
\]
\[
\begin{bmatrix}
	A_r \\ A_\theta \\ A_\phi
\end{bmatrix}
=
\begin{bmatrix}
	\sin\theta\cos\phi & \sin\theta\sin\phi & \cos\theta\\
	\cos\theta\cos\phi & \cos\theta\sin\phi & -\sin\theta\\
	-\sin\phi & \cos\phi  & 0 \\
\end{bmatrix}
\begin{bmatrix}
	A_x \\ A_y \\ A_z
\end{bmatrix}
\]

\extraline
Gradient
\[ \nabla V = \frac{\partial V}{\partial r}\hat{r} + \frac{1}{r}\frac{\partial V}{\partial \theta}\hat{\theta} + \frac{1}{r\sin\theta}\frac{\partial V}{\partial \phi}\hat{\phi} \]

Laplacien
\[ \nabla^2 V = \frac{1}{r^2} \frac{\partial}{\partial r}\left( r^2 \frac{\partial V}{\partial r} \right)
+ \frac{1}{r^2\sin\theta}\frac{\partial}{\partial \theta} \left( \sin\theta \frac{\partial V}{\partial\theta} \right)
+ \frac{1}{r^2\sin^2\theta}\frac{\partial^2 V}{\partial \phi^2} \]

Éléments différentiels
\[ \vv{dl} = dr\,\hat{r} + r\,d\theta\,\hat{\theta} + r\sin\theta\,d\phi\,\hat{\phi} \]
\[\vv{ds} = r^2\sin\theta\,d\theta\,\hat{r} + r\,dr\sin\theta\,d\phi\,\hat{\theta} + r\,dr\,d\theta\,\hat{\phi} \]
\[dv = r^2\,dr\,\sin\theta\,d\theta\,d\phi \]


\hrulefill

\section{Rappel mathématique}

\subsection{Identités trigonométriques}
\halfline
\begin{multicols}{2}
\noindent
\[ \sin^2\theta +  \cos^2\theta = 1 \]
\begin{align*}
\cos 2\theta 
&=\cos^2\theta - \sin^2\theta \\
&=2\cos^2\theta - 1 \\
&=1  - 2\sin^2\theta 
\end{align*}
\[ \sin 2\theta  = 2\sin\theta\cos\theta  \]
\[ \sec^2\theta  = 1+ \tan^2\theta \]
\[ \tan 2\theta = \frac{2\tan\theta}{1-\tan^2\theta} \]
\[ \tan \theta = \pm\sqrt{\frac{1-\cos 2\theta}{1+\cos 2\theta}} \]
\end{multicols}
\[ \sin(A\pm B) = \sin A\cos B \pm \cos A\sin B\]
\[ \cos(A\pm B) = \cos A\cos B \mp \sin A\sin B\]

\subsection{Quelques dérivées}
\halfline
\begin{multicols}{2}
\noindent
\[ \deriv (ax^n) = nax^{n-1}\]
\[ \deriv (e^{ax}) = ae^{ax}\]
\[ \deriv (\ln ax) = \frac{a}{x} \]
\[ \deriv (\sin ax) = a\cos ax\]
\[ \deriv (\cos ax) = -a\sin ax\]
\[ \deriv (\tan ax) = a\sec^2 ax\]
\[ \deriv (\sec x) = \tan x \sec x \]
\[ \deriv (\cot x) = -a\csc^2 ax \]
\[ \deriv (\csc x) = -\cot x \csc x \]
\end{multicols}


\subsection{Intégrales résolus}
\halfline
\begin{multicols}{2}
\noindent
\[ \int x^n \,dx =  \frac{x^{n+ 1}}{n+1} \quad (n\neq 1) \]
\[ \int e^{ax}\,dx = \frac{1}{a}e^{ax} \]
\[ \int \ln(ax) = x\ln(ax) - x \]
\[ \int \sin(ax)\,dx =-\frac{1}{a} \cos(ax) \]
\[ \int \cos(ax)\,dx =\frac{1}{a} \sin(ax) \]
\[ \int \sin^2(ax)\,dx = \frac{x}{2} - \frac{\sin(2ax)}{4a} \]
\[ \int \cos^2(ax)\,dx = \frac{x}{2} + \frac{\sin(2ax)}{4a} \]
%
\[ \int \frac{dx}{x} = \ln \lvert x \rvert \]
\[ \int \frac{dx}{a+bx} = \frac{1}{b}\ln\lvert a+bx\rvert \]
\[ \int \frac{dx}{x^2+a^2} = \frac{1}{a}\arctan\left(\frac{x}{a}\right) \]
\[\int \frac{dx}{\sqrt{x^2+a^2}} = \ln \lvert x + \sqrt{x^2+a^2} \rvert \]
\[ \int \frac{dx}{(x^2+a^2)^{3/2}} = \frac{x}{a^2\sqrt{x^2+a^2}} \]
\[ \int \frac{dx}{(a+bx)^2} = -\frac{1}{b(a+bx)} \]
%
\[ \int \frac{x\,dx}{(x^2+a^2)^{3/2}} = -\frac{1}{\sqrt{x^2+a^2}} \]
\end{multicols}

\[ \int \frac{x\,dx}{(x^2+a^2)^n} = -\frac{1}{2(n-1)} \frac{1}{(x^2+a^2)^{n-1}} \quad (n>0)  \]
\[ \int \frac{x\,dx}{(a^2-bx)^{3/2}} = \frac{2x}{b\sqrt{a^2-bx}} + \frac{4\sqrt{a^2-bx}}{b^2} \]

\[ \int \frac{x^2\,dx}{\sqrt{x^2+a^2}} =  \frac{u}{2}\sqrt{x^2+a^2} - \frac{a^2}{2}\ln(x+\sqrt{x^2+a^2}) \]

\[ \int \frac{x^2\,dx}{(x^2+a^2)^{3/2}} = -\frac{x}{\sqrt{x^2+a^2}} + \ln \lvert x + \sqrt{x^2+a^2} \rvert\]

\hrulefill



\section{Constantes}
\[ \epsilon_0 = \SI{8,85E-12}{\frac{\farad}{\meter}} =  \SI{8,85E-12}{\frac{\coulomb^2}{\newton\cdot\meter^2}} \]
\[ q_e = |e| = \SI{1,6E-19}{\coulomb}
\qquad m_e = \SI{9,11E-31}{\kilogram} \]
\[ \mu_0 = \SI{4\pi E-7}{H/\meter}
\qquad c = \SI{3E8}{\meter/\second} \]

\hrulefill
\scriptsize

\href{https://github.com/zhouluke/PhysicsFormulas}{Formula sheet}  \copyright\ 2021 Luke Zhou \\
\href{http://wch.github.io/latexsheet/}{Template}  \copyright\ 2014 Winston Chang


\end{multicols}
\end{document}
