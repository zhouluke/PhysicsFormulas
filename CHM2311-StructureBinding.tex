% !TeX spellcheck = en_US
\documentclass[10pt,landscape]{article}
\usepackage{multicol}
\usepackage{calc}
\usepackage{ifthen}
\usepackage[landscape]{geometry}
\usepackage{hyperref}
\usepackage{amsmath, amsfonts, amssymb}
\usepackage{mathtools}
\usepackage {graphicx}
\usepackage {textcomp}

\usepackage{siunitx}
\usepackage{physics}

% English spacing
\usepackage{icomma}
\frenchspacing
\usepackage[english]{babel}
 \sisetup{output-decimal-marker = {.}}
 
 
% To make this come out properly in landscape mode, do one of the following
% 1.
%  pdflatex latexsheet.tex
%
% 2.
%  latex latexsheet.tex
%  dvips -P pdf  -t landscape latexsheet.dvi
%  ps2pdf latexsheet.ps



% This sets page margins to .5 inch if using letter paper, and to 1cm
% if using A4 paper. (This probably isn't strictly necessary.)
% If using another size paper, use default 1cm margins.
\ifthenelse{\lengthtest { \paperwidth = 11in}}
	{ \geometry{top=.5in,left=.5in,right=.5in,bottom=.5in} }
	{\ifthenelse{ \lengthtest{ \paperwidth = 297mm}}
		{\geometry{top=1cm,left=1cm,right=1cm,bottom=1cm} }
		{\geometry{top=1cm,left=1cm,right=1cm,bottom=1cm} }
	}

% Turn off header and footer
\pagestyle{empty}
 

% Redefine section commands to use less space
\makeatletter
\renewcommand{\section}{\@startsection{section}{1}{0mm}%
                                {-1ex plus -.5ex minus -.2ex}%
                                {0.5ex plus .2ex}%x
                                {\normalfont\large\bfseries}}
\renewcommand{\subsection}{\@startsection{subsection}{2}{0mm}%
                                {-1explus -.5ex minus -.2ex}%
                                {0.5ex plus .2ex}%
                                {\normalfont\normalsize\bfseries}}
\renewcommand{\subsubsection}{\@startsection{subsubsection}{3}{0mm}%
                                {-1ex plus -.5ex minus -.2ex}%
                                {1ex plus .2ex}%
                                {\normalfont\small\bfseries}}
\makeatother

% Don't print section numbers
\setcounter{secnumdepth}{0}


\setlength{\parindent}{0pt}
\setlength{\parskip}{0pt plus 0.5ex}

\newcommand{\extraline}{\vspace{1em}}
\newcommand{\halfline}{\vspace{0.5em}}
\newcommand{\deriv}{\ensuremath{\frac{d}{dx}}}
\newcommand{\tableindent}{\hspace{1.5em}}
\newcommand{\uvec}[1]{\ensuremath{{\hat{#1}}}}
% -----------------------------------------------------------------------

\begin{document}

\raggedright
\footnotesize
\begin{multicols}{3}

% multicol parameters
% These lengths are set only within the two main columns
%\setlength{\columnseprule}{0.25pt}
\setlength{\premulticols}{1pt}
\setlength{\postmulticols}{1pt}
\setlength{\multicolsep}{1pt}
\setlength{\columnsep}{2pt}

\begin{center}
     \Large{\textbf{Structure \& Bonding}} \\
     \small{Luke Zhou $\cdot$ CHM 2311 $\cdot$ Winter 2022}
\end{center}

\section{Atomic Structure \& Quantum Theory}

Wavelength, frequency and the speed of light
\[ c = \lambda\nu  = \SI{2.998E8}{\meter/\second}  \]

Energy quantization
\[E = \frac{hc}{\lambda} = h\nu  \qquad
 h = \SI{6.626E-34}{\joule\cdot\second}  \]

Energy emitted by a \underline{hydrogen} atom: Bohr model
 \[ \Delta E = E_{h} - E_{l} = R_H \left(  \frac{1}{n_l^2} - \frac{1}{n_h^2} \right) \] 
 \[ R_H =  \SI{2.179E-18}{\joule} \]

Hydrogen emission spectra series \\
\begin{tabular}{@{\tableindent}ll@{}}
	Lyman series & $n_l = 1$ \\
	Balmer series & $n_l = 2$ \\
	Paschen series & $n_l = 3$ \\
	Brackett series & $n_l = 4$ \\
	Pfund series & $n_l = 5$ \\
	Humphreys series & $n_l = 6$ \\
\end{tabular}
\extraline

Wave-Particle Duality  
\[  mv = \frac{h}{\lambda} \]


Heisenberg Uncertainty Principle
\[  \Delta x \Delta p \geq \frac{h}{4\pi} \]

Schrödinger Wave Equation
\[ H\Psi = E\Psi  \]
\[ \left[ \frac{-h^2}{8\pi^2m} +
V(x,y,z)\right] \Psi(x,y,z) = E\Psi(x,y,z)  \]
	
Particle in a box
\[ \Psi(x) = \sqrt{\frac{2}{a}} \sin\left( \frac{n\pi x}{a} \right)  \qquad n \in \mathbb{N} \]
\[ E = \frac{n^2h^2}{8ma^2} \]

Particle on a ring
\[ \Psi(x) = B \exp \left(\frac{in\pi x}{L} \right)
\qquad n \in \mathbb{Z}, L = n\lambda  \]
\[ E = \frac{h^2}{8mL^2} (2n)^2 \]

Hydrogenic wave functions
\[ \Psi(r,\theta,\phi) = R(r) Y(\theta,\phi) \]

\tableindent N.B. $R(r)$ is usually graphed against $r/a_0$


Quantum numbers 
\begin{tabular}{@{\tableindent}lllp{2.5cm}@{}}
$n$ & Principal & $1, 2, 3, 4, \ldots $ & \scriptsize{orbital size} \\
$l$ & Angular & $0, 1, 2 \dots  n-1 $ & \scriptsize{orbital shape} \\
$m_l$ & Magnetic & $0, \pm 1, \pm 2, \ldots \pm l $ & \scriptsize{orbital orientation} \\
$m_s$ & Spin & $-\frac{1}{2}, +\frac{1}{2} $ &  \scriptsize{angular momentum direction} \\
\end{tabular}

\extraline
\tableindent $l$ can also be described with the letters $s, p, d, f, g, \ldots$
\extraline

Radial \& angular nodes

\begin{tabular}{@{\tableindent}llp{2.4cm}@{}}
\# of radial nodes & $n-l-1$ & $R(r) = 0$ \scriptsize{(spheres)}\\
\# of angular nodes & $l$ &  $Y(\theta,\phi)=0$ \scriptsize{(planes, cones)} \\
Total \# of nodes & $n-1$ & 
\end{tabular}
\extraline

Electron configurations

\begin{tabular}{@{\tableindent}p{2.5cm}p{4.5cm}@{}}
Aufbau Principle & Fill lower energy levels first: 1s, 2s, 2p, 3s, 3p, \textbf{4s}, \textbf{3d}, 4p, 5s, 4d, 5p, \ldots\\
Pauli Exclusion Principle & No two e- can have the same set of quantum numbers. \\
Hund's Rule & Fill degenerate levels separately w/ spins parallel -- this minimizes the energies below \\
\end{tabular}
\extraline

Energies to be minimized

\begin{tabular}{@{\tableindent}p{25mm}p{47mm}@{}}
	Coulombic Energy of Repulsion & \textdownarrow{} e- pairing = \textdownarrow{} $E$ \\
	Exchange Energy & \textuparrow{} parallel spins in separate orbitals = \textdownarrow{} $E$
\end{tabular}


Magnetism

\begin{tabular}{@{\tableindent}lp{5cm}@{}}
Paramagnetism & Attracted by a magnetic field; elements w/ unpaired e- \\
Dimagnetism & Slightly repelled by a magnetic field; elements w/ no unpaired e-
\end{tabular}

Transition Metals
\begin{itemize}
\item In ionization, e- are first removed from orbitals w/ density farthest from nucleus; i.e. from $ns$ before $(n-1)s$
\item Half- and fully-filled shells minimize exchange energy \& e- pairing: e.g., Cr ([Ar]4s$^1$3d$^{5}$) \& Cu  ([Ar]4s$^1$3d$^{10}$) 
\item All $M^{2+}$ ions are 3d-only; e.g.,  Fe$^{2+}$ is [Ar]$3d^{6}4s^0$
%\& Fe ([Ar]$4s^23d^{6}$)
\end{itemize}  


Effective nuclear charge
\[ Z_\text{eff} = Z_\text{actual} - S \]

Slater's screening constant \& rules
\begin{enumerate}
\item List orbitals in groups, merging s \& p: (1s) (2s, 2p) (3s, 3p) (3d) (4s, 4p) (4d) (5s, 5p), \ldots 
\item For e- in $ns$ or $np$ orbitals:

	\begin{enumerate}
	\item Each e- in the same $ns$ or $np$ (i.e. same grouping) as the one of interest contributes 0.35 (except 1s: 0.30)
	\item Each in the $(n-1)$ level: 0.85
	\item Each in the $(n-2)$ levels or lower: 1.00
	\end{enumerate}

\item For e- in $nd$ or $nf$ orbitals:

	\begin{enumerate}
	\item Each e- in the same $nd$ or $nf$  orbital as the e- of interest contributes 0.35
	\item Each in the groupings to the left: 1.00
	\end{enumerate}

\end{enumerate}

Periodic trends

\begin{center}
\includegraphics[width=8cm]{./chm2311/periodic-table-trends.png}
\end{center}
 
\begin{itemize} 
\item Atomic radius: size of outermost orbital containing e-. Larger $n$ or higher $Z_\text{eff}$ = lower radius.
\item Ionization energy: for removal of an e-. Always endothermic. Larger atomic radius  = lower IE.  Second IE is much higher than the first.
\item Electron affinity: energy in adding an e- to an isolated atom. Usually exothermic (i.e. negative). Larger atomic radius = less attraction to nucleus = less-negative EA. Second EA is always endothermic. 
\item Electronegativity: ability of a bonded atom to attracted shared electrons. EN is relative.
\end{itemize}

\hrulefill


\section{Constants}

Planck's constant
\[ h = \SI{6.626E-34}{\joule\cdot\second} \]
 
 Rydberg constant for hydrogen
 \[ R_H =  \SI{2.179E-18}{\joule} \]
 
Electron mass
\[  m_e = \SI{9,11E-31}{\kilogram} \]

Speed of light
\[ c = \SI{2.998E8}{\meter/\second}  \]

Bohr radius \scriptsize{(most probable dist b/w nucleus \& e- in a ground-state H atom)}
\[ a_0 = \SI{52.9}{\pico\metre} \]

Conversions
\[ \SI{1}{\metre} = 10^9 \, \si{\nano\metre} = 10^{12} \, \si{\pico\metre}\]
\[ \SI{1}{\joule} = \SI{1}{\kilogram\cdot\metre^2/\sec^2}  \]


\hrulefill

\section{Molecular Structure \& Bonding}

Types of bonding 

\begin{tabular}{@{\tableindent}llp{26mm}<{\raggedright}@{}}
	Covalent & $\text{EN}_A \sim  \text{EN}_B$ & Quantum (wave) mechanics \\
	Polar covalent & $\text{EN}_A < \text{EN}_B$ & \\
	Ionic & $\text{EN}_A \ll  \text{EN}_B$ &  Classic electrostatics \\
\end{tabular}

Lewis structures: formal charge
\[
\begin{pmatrix}
	 \text{\# of valence e-} \\
	 \text{in a free atom of} \\
	  \text{the element}
\end{pmatrix}
-
\begin{pmatrix}
	\text{\# of unshared} \\
	\text{e- on the atom} 
\end{pmatrix}
-
\begin{pmatrix}
	\text{\# of bonds} \\
	\text{to the atom} 
\end{pmatrix}
\]
\[ \text{Total charge on molecule or ion} = \text{sum of formal charges} \]

\extraline

Lewis structures: chemical properties

\begin{tabular}{@{\tableindent}lp{2.45cm}<{\raggedright}p{2.6cm}<{\raggedright}@{}}
Lewis base & Nucleophile (e- pair donor) & -ve formal charges \& lone pairs \\
Lewis acid & Electrophile (e- pair acceptor) & atoms lacking full octets
\end{tabular}
\extraline

VSEPR notation
\[ \text{AX}_n\text{E}_m \]
\tableindent (A: central atom, $n$: \# of bonding pairs, $m$: \# of lone pairs) \\
\extraline


VSEPR (Valence Shell Electron Pair Repulsion Theory): geometries \& shapes

\begin{tabular}{@{}cp{0.35\linewidth}<{\raggedright}p{0.4\linewidth}<{\raggedright}@{}}
Steric \# & Geometry & Shapes \\ \hline
2 & Linear (180°) 
	& Linear \\
3 & Trigonal Planar (120°)
	& Trigonal Planar, Bent / V-shaped \\
4 & Tetrahedral (109.5°)
	& Tetrahedral, Trigonal Pyramidal, Bent / V-shaped  \\
5 & Trigonal Bipyramidal (120°, 90°)
	& Trigonal Bipyramidal, See-saw, T-shaped, Linear  \\
6 & Octahedral (90°)
	& Octahedral, Square Pyramidal, Square Planar \\
7 & Pentagonal Bipyramidal & \\ %(72°, 90°)
8 & Square Antiprismatic & \\ %(70,5°, 99,6°, 109,5°)
\end{tabular}
\extraline

Lone pair placement:  $\downarrow$ \# of nearest neighbours (in terms of angle)
\begin{itemize}
	\item Trigonal bipyramidal: LP occupy equatorial slots (so it has 2 axial nearest neighbours, 90° away) rather than axial (3 equatorial neighbours, 90° away)
	\begin{itemize}	
		\item Lower-EN atoms tend to prefer equatorial positions too
	\end{itemize}
	\item Octahedral: LP occupy axial slots (all neighbours 90° away)
\end{itemize}


VSEPR: degree of repulsion
\[ \text{BP/BP} < \text{BP/LP} < \text{LP/LP} \]
\[ \text{single bond} < \text{double bond} < \text{triple bond} < \text{lone pair} \]

VSEPR: bond angle trends for molecules with \underline{lone pairs}
\begin{itemize}
\item General principle: 
	\begin{itemize}
		\item  $\downarrow$ e- density on A = $\uparrow$ space occupied by LPs = $\downarrow$ repulsion b/w BP 
		\item $\therefore$ $\downarrow$ bond angles
	\end{itemize}

\item Central atom A: 
	\begin{itemize}
		\item $\downarrow$ EN  = $\downarrow$ e- density on A = $\downarrow$ bond angles
		\item $\uparrow$ size =  $\downarrow$ e- density on A = $\downarrow$ bond angles
		%\item $\uparrow$ bond length
	\end{itemize}

\item Terminal atoms X: 
	\begin{itemize}
		\item $\uparrow$ EN  = $\downarrow$ e- density on A = $\downarrow$ bond angles
		\item $\downarrow$ size = $\downarrow$ space needed by X = $\downarrow$ bond angles
		%\item $\downarrow$ bond length
	\end{itemize}
\end{itemize}

Dipole moments \\
\tableindent Direction: $\delta + \rightarrow \delta -$ (less EN $\rightarrow$ more EN) 
\extraline

Valence Bond Theory

\begin{tabular}{@{\tableindent}lp{16mm}p{42mm}@{}}
$\sigma$ bonds & single bonds & head-on overlap of hybridized orbitals \\
$\pi$ bonds & double/triple bonds & interactions b/w  remaining unhybridized orbitals \\
\end{tabular}
\extraline

Valence Bond Theory: Orbital hybridization 
%
\begin{tabular}{@{}cll@{}}
	Steric \# & Geometry & Hybridization \\ \hline
	2 & Linear & (none) \\
	3 & Trigonal Planar 
	& $sp$ \\
	4 & Tetrahedral 
	& $sp^2$ \\
	5 & Trigonal Bipyramidal 
	& $sp^3$ \\
	6 & Octahedral 
	& $sp^3d$ \\
	7 & Pentagonal Bipyramidal & $sp^3d^2$ \\ 
	8 & Square Antiprismatic & $sp^3d^3$ \\
\end{tabular}
\extraline


\hrulefill

\section{Symmetry \& Group Theory}

Symmetry elements
\tableindent Points, axes, planes

Symmetry operations
%
\renewcommand{\arraystretch}{1.4}
\begin{tabular}{@{\tableindent}cp{0.15\linewidth}<{\raggedright}p{0.6\linewidth}<{\raggedright}@{}}
	$E$ & Identity & No change \\
	$C_n$ & (Proper) rotation  & Rotation by $(360/n)$° about an axis of rotation \\
	$\sigma$ & Reflection  & Reflection through a mirror plane \\
	$i$ & Inversion  & Every point $(x, y, z) \rightarrow (-x, -y, -z)$ \\ %-- movement through the centre of the object to a position opposite and as far from the point as initially\\
	$S_n$ & Rotation-reflection (improper rotation)  & Rotation by $(360/n)$° followed by reflection through a plane \underline{perpendicular }to the axis of rotation
\end{tabular}
\renewcommand{\arraystretch}{1}

Types of reflection operations
%
\renewcommand{\arraystretch}{1.4}
\begin{tabular}{@{\tableindent}cp{0.15\linewidth}<{\raggedright}p{0.6\linewidth}<{\raggedright}@{}}
	$\sigma$ & Reflection & Reflection through a mirror plane (general symbol) \\
	$\sigma_h$ & ``Horizontal"  & Reflection through a mirror plane \underline{$\perp$ to the principle axis of rotation} \\
	$\sigma_v$ & ``Vertical" & Reflection through a mirror plane that \underline{includes the principle axis of rotation} \\
	$\sigma_d$ & ``Dihedral" & Reflection through a mirror plane that \underline{bisects two $C_n'$ axes} \\
\end{tabular}
\renewcommand{\arraystretch}{1}

\hrulefill


\scriptsize

\href{https://github.com/zhouluke/PhysicsFormulas}{Typography}  \copyright\ 2022 Luke Zhou. \\
Based on the course CHM 2311 Winter 2022 at the University of Ottawa \\
\href{http://wch.github.io/latexsheet/}{Document template}  \copyright\ 2014 Winston Chang.


\end{multicols}
\end{document}
